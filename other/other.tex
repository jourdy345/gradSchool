\title{Graduate School Pre-exam Solution}
\author{Daeyoung Lim}

\documentclass[answers]{exam}
\usepackage[left=3cm,right=3cm,top=3.5cm,bottom=2cm]{geometry}
\usepackage{amssymb,amsmath}
\usepackage{mathtools}
\usepackage{graphicx}
\usepackage{kotex}
\usepackage[utf8]{inputenc}
\usepackage[T1]{fontenc}
\usepackage{lmodern}
% \usepackage{enumerate}
\usepackage{listings}
\usepackage{courier}
\usepackage{cancel}
\usepackage{array}
\usepackage{courier}
\usepackage{booktabs}
\usepackage{titlesec}
\usepackage[shortlabels]{enumitem}
\usepackage{setspace}
\usepackage{empheq}
\usepackage{tikz}
\usepackage{listings}

% \usepackage[toc,page]{appendix}

\setlength{\heavyrulewidth}{1.5pt}
\setlength{\abovetopsep}{4pt}

\DeclarePairedDelimiter{\ceil}{\lceil}{\rceil}
\newcommand\encircle[1]{%
  \tikz[baseline=(X.base)] 
    \node (X) [draw, shape=circle, inner sep=0] {\strut #1};}
 
% Command "alignedbox{}{}" for a box within an align environment
% Source: http://www.latex-community.org/forum/viewtopic.php?f=46&t=8144
\newlength\dlf  % Define a new measure, dlf
\newcommand\alignedbox[2]{
% Argument #1 = before & if there were no box (lhs)
% Argument #2 = after & if there were no box (rhs)
&  % Alignment sign of the line
{
\settowidth\dlf{$\displaystyle #1$}  
    % The width of \dlf is the width of the lhs, with a displaystyle font
\addtolength\dlf{\fboxsep+\fboxrule}  
    % Add to it the distance to the box, and the width of the line of the box     ㅊ
\hspace{-\dlf}  
    % Move everything dlf units to the left, so that & #1 #2 is aligned under #1 & #2
\boxed{#1 #2}
    % Put a box around lhs and rhs
}
}
\setcounter{secnumdepth}{4}
\lstset{
         basicstyle=\footnotesize\ttfamily, % Standardschrift
         %numbers=left,               % Ort der Zeilennummern
         numberstyle=\tiny,          % Stil der Zeilennummern
         %stepnumber=2,               % Abstand zwischen den Zeilennummern
         numbersep=5pt,              % Abstand der Nummern zum Text
         tabsize=2,                  % Groesse von Tabs
         extendedchars=true,         %
         breaklines=true,            % Zeilen werden Umgebrochen
         keywordstyle=\color{red},
            frame=b,         
 %        keywordstyle=[1]\textbf,    % Stil der Keywords
 %        keywordstyle=[2]\textbf,    %
 %        keywordstyle=[3]\textbf,    %
 %        keywordstyle=[4]\textbf,   \sqrt{\sqrt{}} %
         stringstyle=\color{white}\ttfamily, % Farbe der String
         showspaces=false,           % Leerzeichen anzeigen ?
         showtabs=false,             % Tabs anzeigen ?
         xleftmargin=17pt,
         framexleftmargin=17pt,
         framexrightmargin=5pt,
         framexbottommargin=4pt,
         %backgroundcolor=\color{lightgray},
         showstringspaces=false      % Leerzeichen in Strings anzeigen ?        
 }
 \lstloadlanguages{% Check Dokumentation for further languages ...
         %[Visual]Basic
         %Pascal
         %C
         %C++
         %XML
         %HTML
         Java
 }
    %\DeclareCaptionFont{blue}{\color{blue}} 

\definecolor{myblue}{RGB}{72, 165, 226}
\definecolor{myorange}{RGB}{222, 141, 8}
\titleformat{\paragraph}
{\normalfont\normalsize\bfseries}{\theparagraph}{1em}{}
\titlespacing*{\paragraph}
{0pt}{3.25ex plus 1ex minus .2ex}{1.5ex plus .2ex}
\setlength{\heavyrulewidth}{1.5pt}
\setlength{\abovetopsep}{4pt}
\setlength{\parindent}{0mm}
\linespread{1.3}
\DeclareMathOperator{\sech}{sech}
\DeclareMathOperator{\csch}{csch}
\DeclareMathOperator*{\argmin}{\arg\!\min}
\DeclareMathOperator{\Tr}{Tr}

\newcommand{\bs}{\boldsymbol}
\newcommand{\opn}{\operatorname}
%%%%%%%%%%%%%%%%%%%%%%%%%%%%%%%%%%%%%%%%%%%%%%%%%%%%%%%
% % We use newtheorem to define theorem-like structures
% %
% % Here are some common ones. . .
%%%%%%%%%%%%%%%%%%%%%%%%%%%%%%%%%%%%%%%%%%%%%%%%%%%%%%%
\newtheorem{theorem}{Theorem}
\newtheorem{lemma}{Lemma}
\newtheorem{proposition}{Proposition}
\newtheorem{scolium}{Scolium}   %% And a not so common one.
\newtheorem{definition}{Definition}
\newenvironment{proof}{{\sc Proof:}}{~\hfill QED}
\newenvironment{AMS}{}{}
\newenvironment{keywords}{}{}
%%%%%%%%%%%%%%%%%%%%%%%%%%%%%%%%%%%%%%%%%%%%%%%%%%%%%%%
% %   The first thanks indicates your affiliation
% %
% %  Just the name here.
% %
% % Your mailing address goes at the end.
% %
% % \thanks is also how you indicate grant support
% %
%%%%%%%%%%%%%%%%%%%%%%%%%%%%%%%%%%%%%%%%%%%%%%%%%%%%%%%


\begin{document}
\setstretch{1.5} %줄간격 조정
\newpage
\firstpageheader{}{}{\bf\large Daeyoung Lim \\ Grad School \\ Other}
\runningheader{Daeyoung Lim}{Graduate School Pre-exam}{Other}
\begin{questions}
   \question
   \emph{(2015 late \#2)} $X_{1},\ldots, X_{n}$이 평균이 $\theta$인 지수분포에서 뽑은 임의표본이라고 하자.
   \begin{enumerate}[(a)]
    \item 이 지수분포의 분산에 대한 최대가능도 추정량(MLE)의 점근적 분산(asymptotic variance)을 구하시오.
    \item $n=30$이고 표본평균이 $26.5$라고 할 때, $\mathrm{Pr}\left(X>10\right)$의 최대가능도 추정량(MLE)을 이용하여 $95\%$ 근사 신뢰구간을 구하시오.
    \item 이 분포로부터 $20$에서 중도 절단된(right-censored) 크기 $5$인 표본 $7, 12, 15,20,20$이 주어졌다고 할 때, $\theta$의 최대가능도 추정치를 구하시오.
   \end{enumerate}
   \begin{solution}
    \begin{enumerate}[(a)]
      \item MLE의 점근분포는 다음과 같다.
      \begin{equation}
        \widehat{\theta} \xrightarrow{d}\mathcal{N}\left(\theta,\mathcal{I}\left(\theta\right)^{-1}\right)
      \end{equation}
      따라서 정보량을 구하면
      \begin{equation}
        \mathrm{I}\left(\theta\right) = \theta^{-2}
      \end{equation}
      따라서
      \begin{equation}
        \widehat{\theta}\xrightarrow{d}\mathcal{N}\left(\theta,n\theta^{-2}\right)
      \end{equation}
      \item 먼저 확률을 구하자.
      \begin{align}
        \mathrm{Pr}\left(X>10\right) &= \int_{10}^{\infty}\dfrac{1}{\theta}\exp\left(-\dfrac{x}{\theta}\right)\\
        &= \exp\left(-\dfrac{10}{\theta}\right)\\
        \widehat{\mathrm{Pr}\left(X>10\right)} &= \exp\left(-\dfrac{10}{\widehat{\theta}}\right),\qquad \text{(By the invariance property)}
      \end{align}
      \emph{Delta method}를 사용하면 다음과 같다.
      \begin{equation}
        \sqrt{n}\left(g\left(\widehat{\theta}\right)-g\left(\theta\right)\right)\xrightarrow{d}\mathcal{N}\left(0,\sigma^{2}\left[g'\left(\theta\right)\right]^{2}\right)
      \end{equation}
      따라서 $g\left(\widehat{\theta}\right)=\widehat{\mathrm{Pr}\left(X>10\right)}$이므로
      \begin{equation}
        \sqrt{n}\left(g\left(\widehat{\theta}\right)-g\left(\theta\right)\right)\xrightarrow{d}\mathcal{N}\left(0,n\theta^{-2}\cdot\dfrac{10}{\theta^{2}}\exp\left(-\dfrac{10}{\theta}\right)\right)
      \end{equation}
    \end{enumerate}
   \end{solution}
   \question
   \emph(2016 \#1) 
\end{questions}
\end{document}
