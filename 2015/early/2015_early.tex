\title{Graduate School Pre-exam Solution}
\author{Daeyoung Lim}

\documentclass[answers]{exam}
\usepackage[left=3cm,right=3cm,top=3.5cm,bottom=2cm]{geometry}
\usepackage{amssymb,amsmath}
\usepackage{mathtools}
\usepackage{graphicx}
\usepackage{kotex}
\usepackage[utf8]{inputenc}
\usepackage[T1]{fontenc}
\usepackage{lmodern}
% \usepackage{enumerate}
\usepackage{listings}
\usepackage{courier}
\usepackage{cancel}
\usepackage{array}
\usepackage{courier}
\usepackage{booktabs}
\usepackage{titlesec}
\usepackage[shortlabels]{enumitem}
\usepackage{setspace}
\usepackage{empheq}
\usepackage{tikz}
\usepackage{listings}

% \usepackage[toc,page]{appendix}

\setlength{\heavyrulewidth}{1.5pt}
\setlength{\abovetopsep}{4pt}

\DeclarePairedDelimiter{\ceil}{\lceil}{\rceil}
\newcommand\encircle[1]{%
  \tikz[baseline=(X.base)] 
    \node (X) [draw, shape=circle, inner sep=0] {\strut #1};}
 
% Command "alignedbox{}{}" for a box within an align environment
% Source: http://www.latex-community.org/forum/viewtopic.php?f=46&t=8144
\newlength\dlf  % Define a new measure, dlf
\newcommand\alignedbox[2]{
% Argument #1 = before & if there were no box (lhs)
% Argument #2 = after & if there were no box (rhs)
&  % Alignment sign of the line
{
\settowidth\dlf{$\displaystyle #1$}  
    % The width of \dlf is the width of the lhs, with a displaystyle font
\addtolength\dlf{\fboxsep+\fboxrule}  
    % Add to it the distance to the box, and the width of the line of the box     ㅊ
\hspace{-\dlf}  
    % Move everything dlf units to the left, so that & #1 #2 is aligned under #1 & #2
\boxed{#1 #2}
    % Put a box around lhs and rhs
}
}
\setcounter{secnumdepth}{4}
\lstset{
         basicstyle=\footnotesize\ttfamily, % Standardschrift
         %numbers=left,               % Ort der Zeilennummern
         numberstyle=\tiny,          % Stil der Zeilennummern
         %stepnumber=2,               % Abstand zwischen den Zeilennummern
         numbersep=5pt,              % Abstand der Nummern zum Text
         tabsize=2,                  % Groesse von Tabs
         extendedchars=true,         %
         breaklines=true,            % Zeilen werden Umgebrochen
         keywordstyle=\color{red},
            frame=b,         
 %        keywordstyle=[1]\textbf,    % Stil der Keywords
 %        keywordstyle=[2]\textbf,    %
 %        keywordstyle=[3]\textbf,    %
 %        keywordstyle=[4]\textbf,   \sqrt{\sqrt{}} %
         stringstyle=\color{white}\ttfamily, % Farbe der String
         showspaces=false,           % Leerzeichen anzeigen ?
         showtabs=false,             % Tabs anzeigen ?
         xleftmargin=17pt,
         framexleftmargin=17pt,
         framexrightmargin=5pt,
         framexbottommargin=4pt,
         %backgroundcolor=\color{lightgray},
         showstringspaces=false      % Leerzeichen in Strings anzeigen ?        
 }
 \lstloadlanguages{% Check Dokumentation for further languages ...
         %[Visual]Basic
         %Pascal
         %C
         %C++
         %XML
         %HTML
         Java
 }
    %\DeclareCaptionFont{blue}{\color{blue}} 

\definecolor{myblue}{RGB}{72, 165, 226}
\definecolor{myorange}{RGB}{222, 141, 8}
\titleformat{\paragraph}
{\normalfont\normalsize\bfseries}{\theparagraph}{1em}{}
\titlespacing*{\paragraph}
{0pt}{3.25ex plus 1ex minus .2ex}{1.5ex plus .2ex}
\setlength{\heavyrulewidth}{1.5pt}
\setlength{\abovetopsep}{4pt}
\setlength{\parindent}{0mm}
\linespread{1.3}
\DeclareMathOperator{\sech}{sech}
\DeclareMathOperator{\csch}{csch}
\DeclareMathOperator*{\argmin}{\arg\!\min}
\DeclareMathOperator{\Tr}{Tr}

\newcommand{\bs}{\boldsymbol}
\newcommand{\opn}{\operatorname}
%%%%%%%%%%%%%%%%%%%%%%%%%%%%%%%%%%%%%%%%%%%%%%%%%%%%%%%
% % We use newtheorem to define theorem-like structures
% %
% % Here are some common ones. . .
%%%%%%%%%%%%%%%%%%%%%%%%%%%%%%%%%%%%%%%%%%%%%%%%%%%%%%%
\newtheorem{theorem}{Theorem}
\newtheorem{lemma}{Lemma}
\newtheorem{proposition}{Proposition}
\newtheorem{scolium}{Scolium}   %% And a not so common one.
\newtheorem{definition}{Definition}
\newenvironment{proof}{{\sc Proof:}}{~\hfill QED}
\newenvironment{AMS}{}{}
\newenvironment{keywords}{}{}
%%%%%%%%%%%%%%%%%%%%%%%%%%%%%%%%%%%%%%%%%%%%%%%%%%%%%%%
% %   The first thanks indicates your affiliation
% %
% %  Just the name here.
% %
% % Your mailing address goes at the end.
% %
% % \thanks is also how you indicate grant support
% %
%%%%%%%%%%%%%%%%%%%%%%%%%%%%%%%%%%%%%%%%%%%%%%%%%%%%%%%


\begin{document}
\setstretch{1.5} %줄간격 조정
\newpage
\firstpageheader{}{}{\bf\large Daeyoung Lim \\ Grad School \\ 2015, early}
\runningheader{Daeyoung Lim}{Graduate School Pre-exam}{Year of 2015, early}
\begin{questions}
   \question
   $\left(X,Y\right)$가 이변량 정규분포 $\mathcal{N}_{2}\left(\mu_{1},\mu_{2},\sigma_{1}^{2},\sigma_{2}^{2},\rho\right)$를 따른다고 하자. 즉, $\left(X,Y\right)$의 결합확률분포는
   \begin{equation}
    p\left(x,y\right) =\dfrac{1}{2\pi\sigma_{1}\sigma_{2}\sqrt{1-\rho^{2}}}\exp\left[-\dfrac{1}{2\left(1-\rho^{2}\right)}\left\{\left(\dfrac{x-\mu_{1}}{\sigma_{1}}\right)^{2}-2\rho\dfrac{\left(x-\mu_{1}\right)}{\sigma_{1}}\dfrac{\left(y-\mu_{2}\right)}{\sigma_{2}}+\left(\dfrac{y-\mu_{2}}{\sigma_{2}}\right)^{2}\right\}\right]
   \end{equation}
   이다.
   \begin{enumerate}[(a)]
    \item (10점) $Y=y$로 주어졌을 때, $X$의 조건부 분포가
    \begin{equation}
      \mathcal{N}_{1}\left(\mu_{1}+\rho\dfrac{\sigma_{1}}{\sigma_{2}}\left(y-\mu_{2}\right),\sigma_{1}^{2}\left(1-\rho^{2}\right)\right)
    \end{equation}
    이 됨을 보여라. 단, $\mathcal{N}_{1}\left(\mu,\sigma^{2}\right)$은 평균이 $\mu$이고 표준편차가 $\sigma$인 일변량 정규분포를 나타낸다. (힌트: $Y$의 주변확률분포는 $\mathcal{N}_{1}\left(\mu_{2},\sigma_{2}^{2}\right)$이다.)
    \item (5점) $Y$를 사용할 때, $X$에 대한 best predictor를 쓰시오.
    \item (10점) $\mathrm{Var}\left(X\right), \mathrm{E}\left(\mathrm{Var}\left(X\,|\,Y\right)\right)$ 및 $\mathrm{Var}\left(\mathrm{E}\left(X\,|\,Y\right)\right)$를 구하고 이들 간의 관계식을 쓰시오.
   \end{enumerate}
   \begin{solution}
    \begin{enumerate}[(a)]
      \item 먼저 일반적인 다변량 정규분포에서 해보자. 즉 어떤 random vector $\mathbf{X}\sim\mathcal{N}\left(\bs{\mu},\bs{\Sigma}\right)$를 때를 때 다음의 파티션을 생각할 수 있다. (모든 다변량 정규분포들의 조건부 분포들은 정규분포라는 정리가 있다. 증명은 생략한다.)
      \begin{align}
        \bs{\mu} &= \begin{bmatrix}\bs{\mu}_{1}&\bs{\mu}_{2}\end{bmatrix}'\\
        \mathbf{X} &= \begin{bmatrix}\mathbf{x}_{1}&\mathbf{x}_{2}\end{bmatrix}'
      \end{align}
      그리고 공분산행렬도 다음처럼 구획할 수 있다.
      \begin{equation}
        \begin{bmatrix}\bs{\Sigma}_{11}&\bs{\Sigma}_{12}\\\bs{\Sigma}_{21} &\bs{\Sigma}_{22} \end{bmatrix}
      \end{equation}
      이때 다음을 정의하자. $\mathbf{z}=\mathbf{x}_{1}+\mathbf{Ax}_{2}$이고 여기서 $\mathbf{A}=-\bs{\Sigma}_{12}\bs{\Sigma}_{22}^{-1}$. 그러면 다음을 알 수 있다.
      \begin{align}
        \mathrm{Cov}\left(\mathbf{z},\mathbf{x}_{2}\right) &= \mathrm{Cov}\left(\mathbf{x}_{1},\mathbf{x}_{2}\right)+\mathrm{Cov}\left(\mathbf{Ax}_{2},\mathbf{x}_{2}\right)\\
        &=\bs{\Sigma}_{12}+\mathbf{A}\mathrm{Var}\left(\mathbf{x}_{2}\right)\\
        &=\bs{\Sigma}_{12}-\bs{\Sigma}_{12}\bs{\Sigma}_{22}^{-1}\bs{\Sigma}_{22}\\
        &= 0
      \end{align}
      그러므로 $\mathbf{z}$와 $\mathbf{x}_{2}$는 uncorrelated이며, 다변량 정규분포는 uncorrelated가 독립임을 의미한다. $\mathrm{E}\left(\mathbf{z}\right)=\bs{\mu}_{1}+\mathbf{A}\bs{\mu}_{2}$임은 자명하며, 따라서 다음을 유도할 수 있다.
      \begin{align}
        \mathrm{E}\left(\mathbf{x}_{1}\,|\,\mathbf{x}_{2}\right) &= \mathrm{E}\left(\mathbf{z}-\mathbf{Ax}_{2}\,|\,\mathbf{x}_{2}\right)\\
        &=\mathrm{E}\left(\mathbf{z}\,|\,\mathbf{x}_{2}\right)-\mathrm{E}\left(\mathbf{Ax}_{2}\,|\,\mathbf{x}_{2}\right)\\
        &= \mathrm{E}\left(\mathbf{z}\right)-\mathbf{Ax}_{2}\\
        &= \bs{\mu}_{1}+\mathbf{A}\left(\bs{\mu}_{2}-\mathbf{x}_{2}\right)\\
        &= \bs{\mu}_{1}+\bs{\Sigma}_{12}\bs{\Sigma}_{22}^{-1}\left(\mathbf{x}_{2}-\bs{\mu}_{2}\right)
      \end{align}
      그리고 조건부 공분산 행렬은 다음과 같다.
      \begin{align}
        \mathrm{Var}\left(\mathbf{x}_{1}\,|\,\mathbf{x}_{2}\right) &= \mathrm{Var}\left(\mathbf{z}-\mathbf{Ax}_{2}\,|\,\mathbf{x}_{2}\right)\\
        &= \mathrm{Var}\left(\mathbf{z}\,|\,\mathbf{x}_{2}\right)+\mathrm{Var}\left(\mathbf{Ax}_{2}\,|\,\mathbf{x}_{2}\right)-\mathbf{A}\mathrm{Cov}\left(\mathbf{z},-\mathbf{x}_{2}\right)-\mathrm{Cov}\left(\mathbf{z},-\mathbf{x}_{2}\right)\mathbf{A}'\\
        &= \mathrm{Var}\left(\mathbf{z}\,|\,\mathbf{x}_{2}\right)\\
        &= \mathrm{Var}\left(\mathbf{z}\right)
      \end{align}
      따라서
      \begin{align}
        \mathrm{Var}\left(\mathbf{x}_{1}\,|\,\mathbf{x}_{2}\right) &= \mathrm{Var}\left(\mathbf{z}\right)\\
        &=\mathrm{Var}\left(\mathbf{x}_{1}+\mathbf{Ax}_{2}\right)\\
        &=\mathrm{Var}\left(\mathbf{x}_{1}\right)+\mathbf{A}\mathrm{Var}\left(\mathbf{x}_{2}\right)\mathbf{A}'+\mathbf{A}\mathrm{Cov}\left(\mathbf{x}_{1},\mathbf{x}_{2}\right)+\mathrm{Cov}\left(\mathbf{x}_{2},\mathbf{x}_{1}\right)\mathbf{A}'\\
        &= \bs{\Sigma}_{11}+\bs{\Sigma}_{12}\bs{\Sigma}_{22}^{-1}\bs{\Sigma}_{22}\bs{\Sigma}_{22}^{-1}\bs{\Sigma}_{21}-2\bs{\Sigma}_{12}\bs{\Sigma}_{22}^{-1}\bs{\Sigma}_{21}\\
        &= \bs{\Sigma}_{11}+\bs{\Sigma}_{12}\bs{\Sigma}_{22}^{-1}\bs{\Sigma}_{21}-2\bs{\Sigma}_{12}\bs{\Sigma}_{22}^{-1}\bs{\Sigma}_{21}\\
        &=\bs{\Sigma}_{11}-\bs{\Sigma}_{12}\bs{\Sigma}_{22}^{-1}\bs{\Sigma}_{21}
      \end{align}
      조건부 공분산행렬의 꼴을 $\bs{\Sigma}$의 $\bs{\Sigma}_{22}$에 대한 \emph{Schur complement}라고 한다.\par
      이 문제에서처럼 이변량으로 축소하면 $\mathbf{X}=\begin{bmatrix}X&Y\end{bmatrix}'$, $\bs{\mu}=\begin{bmatrix}\mu_{1}&\mu_{2}\end{bmatrix}'$, 그리고 공분산행렬이 다음과 같아진다.
      \begin{equation}
        \bs{\Sigma} = \begin{bmatrix}\sigma_{1}^{2}&\rho\sigma_{1}\sigma_{2}\\\rho\sigma_{1}\sigma_{2}&\sigma_{2}^{2}\end{bmatrix}
      \end{equation}
      그러므로
      \begin{align}
        \mathrm{E}\left(X\,|\,Y=y\right) &= \mu_{1}+\rho\dfrac{\sigma_{1}}{\sigma_{2}}\left(y-\mu_{2}\right)\\
        \mathrm{Var}\left(X\,|\,Y=y\right) &= \sigma_{1}^{2}-\rho^{2}\sigma_{1}^{2}\\
        &= \sigma_{1}^{2}\left(1-\rho^{2}\right)
      \end{align}
      \item MSE의 관점에서 $Y$의 정보를 알고 있을 때, 즉 $Y$가 주어졌을 때 MSE를 최소화하는 $Y$의 함수를 찾고자 한다. 즉
      \begin{equation}
        g^{*} = \argmin_{g}\mathrm{E}\left(\left(X-g\left(Y\right)\right)^{2}\middle| Y\right)
      \end{equation}
      그럴 때 $g^{*}=\mathrm{E}\left(X\,|\,Y\right)$가 된다. 따라서 위와 같은 이변량 정규분포라면 조건부 평균이 가장 좋은 예측값이다.
      \begin{equation}
        g^{*} =\mu_{1}+\rho\dfrac{\sigma_{1}}{\sigma_{2}}\left(Y-\mu_{2}\right)
      \end{equation}
      \item \emph{Variance decomposition}이다. 조건부 두 개 더하면 $X$의 분산이 나온다.
      \begin{align}
        \mathrm{E}\left(\mathrm{Var}\left(X\,|\,Y\right)\right) &= \sigma_{1}^{2}\left(1-\rho^{2}\right)\\
        \mathrm{Var}\left(\mathrm{E}\left(X\,|\,Y\right)\right) &= \rho^{2}\dfrac{\sigma_{1}^{2}}{\sigma_{2}^{2}}\mathrm{Var}\left(Y\right)\\
        &=\sigma_{1}^{2}\rho^{2}\\
        \mathrm{Var}\left(X\right) &= \mathrm{E}\left(\mathrm{Var}\left(X\,|\,Y\right)\right) + \mathrm{Var}\left(\mathrm{E}\left(X\,|\,Y\right)\right)
      \end{align}
    \end{enumerate}
   \end{solution}
   \question
   Let $X_{1},\ldots,X_{n}$ be a random sample from the distribution with the following probability density function,
   $$
    f\left(x;\theta\right) = \begin{cases}\dfrac{4y^{3}}{\theta^{4}},&0\leq y\leq\theta,\; \theta>0\\ 0,&\text{elsewhere}\end{cases}
   $$
   \begin{enumerate}[(a)]
    \item Find the MLE $\widehat{\theta}_{1n}$ of $\theta$ and compute the MSE (mean squared error) of $\widehat{\theta}_{1n}$.
    \item Find an unbiased estimator of $\widehat{\theta}_{2n}$ of $\theta$ based on the sample average $\overline{X}_{n}$.
    \item Which one would you like better between $\widehat{\theta}_{1n}$ and $\widehat{\theta}_{2n}$ as a point estimator of $\theta$? Give your reasoning.
   \end{enumerate}
   \begin{solution}

   \end{solution}
   \question
   $X_{1},\ldots,X_{n}$가 다음의 결합 확률밀도함수를 갖는 다항분포를 따른다고 하자.
   $$
    f\left(x_{1},\ldots,x_{n}\right) = {{n}\choose{x_{1}\cdots x_{k}}}p_{1}^{x_{1}}\cdots p_{k}^{x_{k}},\quad x_{i}=0,\ldots, n\;\left(i=1,\ldots,k\right),\;x_{1}+\cdots+x_{k}=n
   $$
   단, $p_{1}+\cdots+p_{k}=1$이다.
   $$
    H_{0}:p_{i}=p_{i0},\; i=1,\ldots,k \quad \text{대} \quad H_{a}:\text{not $H_{0}$}
   $$
   의 가설검정에 대하여 다음에 답하시오.
   \begin{enumerate}[(a)]
    \item 가능도비(또는 일반화 가능도비) 검정의 검정통계량을 구하시오.
    \item $n$의 값이 클 때, 유의수준 $\alpha$인 가능도비 검정의 근사적 기각역을 구하시오.
   \end{enumerate}
   \begin{solution}
    \begin{enumerate}[(a)]
      \item 먼저 최대가능도추정량을 구하자. 로그가능도함수는 다음과 같다.
      \begin{equation}
        \ell\left(p_{1},\ldots,p_{k}\,|\,\left\{X_{i}\right\}_{i=1}^{n}\right)=\ln n!-\sum_{i=1}^{k}\ln X_{i}!+\sum_{i=1}^{k}X_{i}\ln p_{i}
      \end{equation}
      하지만 모수를 추정하는데 모수에 제약조건이 붙어있다. 따라서 라그랑지 승수를 붙여야 한다.
      \begin{equation}
        \ell^{*}\left(p_{1},\ldots,p_{k}\,|\,\left\{X_{i}\right\}_{i=1}^{n}\right)=\ln n!-\sum_{i=1}^{k}\ln X_{i}!+\sum_{i=1}^{k}X_{i}\ln p_{i}+\lambda\left(1-\sum_{i=1}^{k}p_{i}\right)
      \end{equation}
      그러므로 다음 두 편도함수를 구할 수 있다.
      \begin{align}
        \dfrac{\partial}{\partial p_{i}}\ell^{*} &= \dfrac{X_{i}}{p_{i}}-\lambda=0\\
        \dfrac{\partial}{\partial \lambda}\ell^{*} &= 1-\sum_{i=1}^{k}p_{i}=0
      \end{align}
      라그랑지 승수에 대한 편도함수는 원래 제약조건을 다시 돌려준다. 따라서 (37)만을 이용해서 $X_{i}=\lambda p_{i}$이므로
      \begin{equation}
        \sum_{i=1}^{k}X_{k} = \lambda \sum_{i=1}^{k}p_{i}
      \end{equation}
      따라서 $\widehat{\lambda}=n$이 된다. 다시 이를 (37)에 넣으면
      \begin{equation}
        \hat{p}_{i} = \dfrac{X_{i}}{n}
      \end{equation}
      이 나온다. 일반화 가능도비를 구하면
      \begin{align}
        \Lambda &= \dfrac{L_{0}}{\widehat{L}}\\
        &= \prod_{i=1}^{k}\left(\dfrac{p_{i0}}{\hat{p}_{i}}\right)^{X_{i}}\\
        &= \prod_{i=1}^{k}n^{Xi}\left(\dfrac{p_{i0}}{X_{i}}\right)^{X_{i}}
      \end{align}
      $-2\ln\Lambda\xrightarrow{d}\chi^{2}\left(k-1\right)$라고 알려져 있다. 여기서 자유도가 $k-1$인 이유는 제약조건이 1개 있기 때문이다. 그러므로 검정통계량은 $-2\ln\Lambda$이다.
      \item 위의 근사분포를 이용해서 chi-squared distribution의 임계값을 찾으면 된다.
      \item (첨부) (a)에서 정확검정을 할 수도 있다. 즉, 관측한 값을 벡터로 $\mathbf{x}$라 나타내기로 하자. 그러면 유의확률(p-value)는 다음과 같이 brute-force로 구할 수도 있다.
      \begin{equation}
        \text{p-value} = \sum_{\left\{\mathbf{y}\,|\,f\left(\mathbf{y}|H_{0}\right)\leq f\left(\mathbf{x}|H_{0}\right) \right\}}f\left(\mathbf{y}\right)
      \end{equation}
      즉, 관측한 값보다 확률이 작아지는 모든 가능한 outcome $\mathbf{y}$를 구해서 확률값을 다 더하면 된다. 하지만 카테고리의 개수가 많아지고 관측치가 많아질수록 정확검정은 너무 고통스러워진다.
    \end{enumerate}
   \end{solution}
   \question
   A regression analysis (involving 45 observations) relating a dependent variable ($Y$) and two independent variables resulted in the following information.
   $$
    \hat{y} = 0.408+1.3387x_{1}+2x_{2}
   $$
   The SSE for the above model is $49$.\par
   When two other independent variables were added to the model, the following information was provided.
   $$
    \hat{y}=1.2+3.0x_{1}+12x_{2}+4.0x_{3}+8x_{4}
   $$
   This latter model's SSE is $40$.\par
   With $\alpha=0.05$, test to determine if the two added independent variables contribute significantly to the model.
   \begin{solution}

   \end{solution}
\end{questions}
\end{document}
