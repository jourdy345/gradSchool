\title{Graduate School Pre-exam Solution}
\author{Daeyoung Lim}

\documentclass[answers]{exam}
\usepackage[left=3cm,right=3cm,top=3.5cm,bottom=2cm]{geometry}
\usepackage{amssymb,amsmath}
\usepackage{mathtools}
\usepackage{graphicx}
\usepackage{kotex}
\usepackage[utf8]{inputenc}
\usepackage[T1]{fontenc}
\usepackage{lmodern}
% \usepackage{enumerate}
\usepackage{listings}
\usepackage{courier}
\usepackage{cancel}
\usepackage{array}
\usepackage{courier}
\usepackage{booktabs}
\usepackage{titlesec}
\usepackage[shortlabels]{enumitem}
\usepackage{setspace}
\usepackage{empheq}
\usepackage{tikz}
\usepackage{listings}

% \usepackage[toc,page]{appendix}

\setlength{\heavyrulewidth}{1.5pt}
\setlength{\abovetopsep}{4pt}

\DeclarePairedDelimiter{\ceil}{\lceil}{\rceil}
\newcommand\encircle[1]{%
  \tikz[baseline=(X.base)] 
    \node (X) [draw, shape=circle, inner sep=0] {\strut #1};}
 
% Command "alignedbox{}{}" for a box within an align environment
% Source: http://www.latex-community.org/forum/viewtopic.php?f=46&t=8144
\newlength\dlf  % Define a new measure, dlf
\newcommand\alignedbox[2]{
% Argument #1 = before & if there were no box (lhs)
% Argument #2 = after & if there were no box (rhs)
&  % Alignment sign of the line
{
\settowidth\dlf{$\displaystyle #1$}  
    % The width of \dlf is the width of the lhs, with a displaystyle font
\addtolength\dlf{\fboxsep+\fboxrule}  
    % Add to it the distance to the box, and the width of the line of the box     ㅊ
\hspace{-\dlf}  
    % Move everything dlf units to the left, so that & #1 #2 is aligned under #1 & #2
\boxed{#1 #2}
    % Put a box around lhs and rhs
}
}
\setcounter{secnumdepth}{4}
\lstset{
         basicstyle=\footnotesize\ttfamily, % Standardschrift
         %numbers=left,               % Ort der Zeilennummern
         numberstyle=\tiny,          % Stil der Zeilennummern
         %stepnumber=2,               % Abstand zwischen den Zeilennummern
         numbersep=5pt,              % Abstand der Nummern zum Text
         tabsize=2,                  % Groesse von Tabs
         extendedchars=true,         %
         breaklines=true,            % Zeilen werden Umgebrochen
         keywordstyle=\color{red},
            frame=b,         
 %        keywordstyle=[1]\textbf,    % Stil der Keywords
 %        keywordstyle=[2]\textbf,    %
 %        keywordstyle=[3]\textbf,    %
 %        keywordstyle=[4]\textbf,   \sqrt{\sqrt{}} %
         stringstyle=\color{white}\ttfamily, % Farbe der String
         showspaces=false,           % Leerzeichen anzeigen ?
         showtabs=false,             % Tabs anzeigen ?
         xleftmargin=17pt,
         framexleftmargin=17pt,
         framexrightmargin=5pt,
         framexbottommargin=4pt,
         %backgroundcolor=\color{lightgray},
         showstringspaces=false      % Leerzeichen in Strings anzeigen ?        
 }
 \lstloadlanguages{% Check Dokumentation for further languages ...
         %[Visual]Basic
         %Pascal
         %C
         %C++
         %XML
         %HTML
         Java
 }
    %\DeclareCaptionFont{blue}{\color{blue}} 

\definecolor{myblue}{RGB}{72, 165, 226}
\definecolor{myorange}{RGB}{222, 141, 8}
\titleformat{\paragraph}
{\normalfont\normalsize\bfseries}{\theparagraph}{1em}{}
\titlespacing*{\paragraph}
{0pt}{3.25ex plus 1ex minus .2ex}{1.5ex plus .2ex}
\setlength{\heavyrulewidth}{1.5pt}
\setlength{\abovetopsep}{4pt}
\setlength{\parindent}{0mm}
\linespread{1.3}
\DeclareMathOperator{\sech}{sech}
\DeclareMathOperator{\csch}{csch}
\DeclareMathOperator*{\argmin}{\arg\!\min}
\DeclareMathOperator{\Tr}{Tr}

\newcommand{\bs}{\boldsymbol}
\newcommand{\opn}{\operatorname}
%%%%%%%%%%%%%%%%%%%%%%%%%%%%%%%%%%%%%%%%%%%%%%%%%%%%%%%
% % We use newtheorem to define theorem-like structures
% %
% % Here are some common ones. . .
%%%%%%%%%%%%%%%%%%%%%%%%%%%%%%%%%%%%%%%%%%%%%%%%%%%%%%%
\newtheorem{theorem}{Theorem}
\newtheorem{lemma}{Lemma}
\newtheorem{proposition}{Proposition}
\newtheorem{scolium}{Scolium}   %% And a not so common one.
\newtheorem{definition}{Definition}
\newenvironment{proof}{{\sc Proof:}}{~\hfill QED}
\newenvironment{AMS}{}{}
\newenvironment{keywords}{}{}
%%%%%%%%%%%%%%%%%%%%%%%%%%%%%%%%%%%%%%%%%%%%%%%%%%%%%%%
% %   The first thanks indicates your affiliation
% %
% %  Just the name here.
% %
% % Your mailing address goes at the end.
% %
% % \thanks is also how you indicate grant support
% %
%%%%%%%%%%%%%%%%%%%%%%%%%%%%%%%%%%%%%%%%%%%%%%%%%%%%%%%


\begin{document}
\setstretch{1.5} %줄간격 조정
\newpage
\firstpageheader{}{}{\bf\large Daeyoung Lim \\ Grad School \\ Year of 2014, late}
\runningheader{Daeyoung Lim}{Graduate School Pre-exam}{2014 late}
\begin{questions}
   \question
   Let $X$ be a random variable defined on $[0,\theta]$ with the density function
   $$
    f\left(x\right)=1/\theta,\qquad 0\leq x\leq \theta,\;\; 1<\theta \leq 2.
   $$
   Suppose that we only observe $Y=X$ if $0\leq X\leq 1$ and $Y=0$ if $X>1$,
   \begin{enumerate}[(a)]
    \item Obtain the probability density function (PDF) of $Y$ and verify that it is a PDF.
    \item Obtain the cumulative distribution function (CDF) of Y and verify that it is a CDF. Also sketch the cdf.
   \end{enumerate}
   \begin{solution}
    \begin{enumerate}[(a)]
      \item 흔히 연속형 변수 혹은 이산형 변수만 보았는데 여기서 $Y$가 혼합형 변수이다. 즉
      \begin{equation}
        Y=\begin{cases}X,&\text{if $0\leq X\leq 1$}\\0,&\text{if $X>1$}  \end{cases}
      \end{equation}
      인데 다시 말하면
      \begin{equation}
        f_{Y}\left(y\right)=\begin{cases}\mathrm{Pr}\left(0\leq X\leq 1\right),&\text{if $0\leq y\leq 1$}\\ \mathrm{Pr}\left(X>1\right),&\text{if $y=0$} \end{cases}
      \end{equation}
      이다. $0\leq y\leq 1$일 때는 연속형 변수이고, $y=0$일 때는 이산형이다. 연속형일 때도 $y=0$일 수 있어서 범위가 잘못된 것처럼 보일 수 있지만 어차피 연속형일 때 $\mathrm{Pr}\left(Y\in\left\{0\right\}\right)=0$이므로 null set이 되어 상관없다. 따라서 이것이 PDF임을 보이려면 전체 정의역에서 확률이 1이 되는지 보아야 한다. 우선 확률을 구하자.
      \begin{align}
        \mathrm{Pr}\left(0\leq X\leq 1\right) &= \int_{0}^{1}\dfrac{1}{\theta}\,dx\\
        &= \dfrac{1}{\theta}\\
        \mathrm{Pr}\left(X>1\right) &= 1-\dfrac{1}{\theta}\\
        f_{Y}\left(y\right)&=\begin{cases}\dfrac{1}{\theta},&\text{if $0\leq y\leq 1$}\\ 1-\dfrac{1}{\theta},&\text{if $y=0$} \end{cases}
      \end{align}
      따라서
      \begin{align}
        \mathrm{Pr}\left(0\leq Y\leq 1\right)+\mathrm{Pr}\left(Y=0\right)&=\int_{0}^{1}\dfrac{1}{\theta}\,dy + 1-\dfrac{1}{\theta}\\
        &=1
      \end{align}
      \item CDF도 마찬가지이다.
      \begin{align}
        F_{Y}\left(y\right) &= \begin{cases}1-\dfrac{1}{\theta},&\text{if $y=0$}\\\dfrac{y}{\theta}+\left(1-\dfrac{1}{\theta}\right),&\text{if $0<y\leq 1$} \end{cases}
      \end{align}
      개형은 각자.
    \end{enumerate}
   \end{solution}
   \question
   랜덤표본 $X_{1},\ldots, X_{n}$의 누적분포(CDF)에 대한 $F\left(x\right)=\mathrm{Pr}\left(X_{1}\leq x\right)$의 추정량으로 경험적 확률분포(empirical distribution function)인 $F_{n}\left(x\right)=\left(\text{\# of $X_{i}\leq x$}\right)/n$을 고려할 때, $F_{n}\left(x\right)$의 일치성(consistency)을 보여라.
   \begin{solution}
     일치성은 확률수렴(convergence in probability; convergence in measure)이므로 대수의 법칙에 의해 증명할 수 있다. 만약 이것이 \emph{almost-sure convergence}였다면 이를 \emph{Glivenko–Cantelli theorem}이라 부르고 증명이 매우 어려워진다. 아무튼 대수의 법칙에 의해 다음을 알고 있다.
     \begin{equation}
      \dfrac{1}{n}\sum_{i=1}^{n}\delta_{A}\left(X_{i}\right)\xrightarrow{p}\mathrm{Pr}\left(X_{1}\in A\right)
     \end{equation}
     우리가 일반적으로 알고 있는 대수의 법칙과 똑같다. 단지 \emph{dirac-delta function}은 indicator function의 기능을 함으로써 기댓값을 취하면 확률이 될 뿐이다. 즉,
     \begin{equation}
      \delta_{A}\left(x\right) = \begin{cases}1, & \text{if $x\in A$}\\ 0, & \text{if $x \notin A$} \end{cases}
     \end{equation}
    따라서 경험적 확률분포를 다시 쓰면 $A=\left\{X\,|\,X\leq x\text{ for some $x$}\right\}$이라 할 때
     \begin{equation}
      F_{n}\left(x\right)=\dfrac{1}{n}\sum_{i=1}^{n}\delta_{A}\left(X_{i}\right)
     \end{equation}
     이 된다. 고로 똑같이
     \begin{equation}
      \dfrac{1}{n}\sum_{i=1}^{n}\delta_{A}\left(X_{i}\right)\xrightarrow{p}\mathrm{Pr}\left(X_{1}\in A\right)\quad \left(=\mathrm{Pr}\left(X_{1}\leq x\right)\right)
     \end{equation}
     따라서 $F_{n}\left(x\right)\xrightarrow{p}F\left(x\right)$.
   \end{solution}
   \question
   Let $X_{1},\ldots,X_{n}$ be a random sample from a distribution with probability density function $f\left(x;\theta\right)=1/\theta$ if $0\leq x \leq \theta$ and zero otherwise. Derive the likelihood ratio test of size $\alpha$ of $H_{0}:\theta=\theta_{0}$ vs $H_{1}:\theta\neq \theta_{0}$.
   \begin{solution}
    MLE는 $X_{\left(n\right)}$이므로 가능도비는
    \begin{equation}
      \Lambda = \begin{cases}\left(\dfrac{X_{\left(n\right)}}{\theta_{0}}\right)^{n}, & \text{if $X_{\left(n\right)}\leq \theta_{0}$}\\0,&\text{if $X_{\left(n\right)}>\theta_{0}$} \end{cases}
    \end{equation}
    이는 당연한 것이 모수의 참값이 $\theta_{0}$라면 관측치 중 가장 큰 값 $X_{\left(n\right)}$이 $\theta_{0}$보다 절대 커지면 안 된다. 만약 크다면 귀무가설을 기각할 수밖에 없다. 따라서 상식에도 부합한다. 엄밀하게 계산하고 싶으면 indicator function을 이용해서 구하면 된다. 따라서 $\Lambda \leq k$일 때 기각역이므로
    \begin{equation}
      \mathrm{RR}=\left\{X_{\left(n\right)}\,|\, X_{\left(n\right)}\leq \theta_{0}k^{1/n}\quad \text{or}\quad X_{\left(n\right)}>\theta_{0}\right\}
    \end{equation}
    이 된다.
   \end{solution}
   \question
   다음과 같은 단순선형회귀모형
   $$
    Y=\beta_{0}+\beta_{1}X+\epsilon
   $$
   을 고려하자. 여기서, $Y$는 반응변수를, $X$는 설명 변수를, 그리고 $\beta_{0},\beta_{1}$는 회귀 계들을 의미하며 $\epsilon$은 평균이 $0$, 분산이 $\sigma^{2}$인 정규분포를 따른다고 가정하자.
   \begin{enumerate}[(a)]
    \item $\beta_{1}$의 최소제곱추정량(LSE) $\widehat{\beta}_{1}$은 $Y$와 $X$ 사이의 표본상관계수(correlation coefficient) $r$과 동일한 부호를 가진다는 것을 보여라.
    \item 가설 $H_{0}:\beta_{1}=0$ vs $H_{1}:\beta_{1}\neq 0$을 검정하기 위한 t-test 검정통계량은
    $$
      t=\dfrac{\sqrt{n-1}r}{\sqrt{1-r^{2}}}
    $$ 
    로 표현될 수 있음을 보여라.
   \end{enumerate}
   \begin{solution}
    \begin{enumerate}[(a)]
      \item 단순회귀일 때 다음과 같은 관계식이 성립한다.
      \begin{equation}
        r_{xy}=\widehat{\beta}_{1}\dfrac{s_{x}}{s_{y}}
      \end{equation}
      표준편차는 항상 0보다 크므로 부호에 영향을 주지 않는다. 따라서 부호가 같다.
      \item 모상관계수를 $\rho$라고 할 때 $\beta_{1}=0$을 검정하는 것은 $\rho=0$을 검정하는 것과 같다. 따라서 $\rho=0$라는 귀무가설 하에서
      \begin{equation}
        f\left(r\right) = \dfrac{1}{\Gamma\left(1/2\right)}\dfrac{\Gamma\left(\left(n-1\right)/2\right)}{\Gamma\left(\left(n-2\right)/2\right)}\left(1-r^{2}\right)^{\left(n-4\right)/2}
      \end{equation}
      이라는 사실이 알려져 있다. (\emph{Mathematical Methods of Statistics, Harald Cramer} p.400 참조) 이를 변수 변환하면
      \begin{equation}
        t=\dfrac{\sqrt{n-1}r}{\sqrt{1-r^{2}}}\sim t\left(n-2\right)
      \end{equation}
      이다.
    \end{enumerate}
   \end{solution}
\end{questions}
\end{document}
