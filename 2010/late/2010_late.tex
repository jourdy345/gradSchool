\title{Graduate School Pre-exam Solution}
\author{Daeyoung Lim}

\documentclass[answers]{exam}
\usepackage[left=3cm,right=3cm,top=3.5cm,bottom=2cm]{geometry}
\usepackage{amssymb,amsmath}
\usepackage{graphicx}
\usepackage{kotex}
\usepackage[utf8]{inputenc}
\usepackage[T1]{fontenc}
\usepackage{lmodern}
% \usepackage{enumerate}
\usepackage{listings}
\usepackage{courier}
\usepackage{cancel}
\usepackage{array}
\usepackage{courier}
\usepackage{booktabs}
\usepackage{titlesec}
\usepackage[shortlabels]{enumitem}
\usepackage{setspace}
\usepackage{empheq}
\usepackage{tikz}
% \usepackage[toc,page]{appendix}

\setlength{\heavyrulewidth}{1.5pt}
\setlength{\abovetopsep}{4pt}


\newcommand\encircle[1]{%
  \tikz[baseline=(X.base)] 
    \node (X) [draw, shape=circle, inner sep=0] {\strut #1};}
 
% Command "alignedbox{}{}" for a box within an align environment
% Source: http://www.latex-community.org/forum/viewtopic.php?f=46&t=8144
\newlength\dlf  % Define a new measure, dlf
\newcommand\alignedbox[2]{
% Argument #1 = before & if there were no box (lhs)
% Argument #2 = after & if there were no box (rhs)
&  % Alignment sign of the line
{
\settowidth\dlf{$\displaystyle #1$}  
    % The width of \dlf is the width of the lhs, with a displaystyle font
\addtolength\dlf{\fboxsep+\fboxrule}  
    % Add to it the distance to the box, and the width of the line of the box
\hspace{-\dlf}  
    % Move everything dlf units to the left, so that & #1 #2 is aligned under #1 & #2
\boxed{#1 #2}
    % Put a box around lhs and rhs
}
}
\setcounter{secnumdepth}{4}
\lstset{
         basicstyle=\footnotesize\ttfamily, % Standardschrift
         %numbers=left,               % Ort der Zeilennummern
         numberstyle=\tiny,          % Stil der Zeilennummern
         %stepnumber=2,               % Abstand zwischen den Zeilennummern
         numbersep=5pt,              % Abstand der Nummern zum Text
         tabsize=2,                  % Groesse von Tabs
         extendedchars=true,         %
         breaklines=true,            % Zeilen werden Umgebrochen
         keywordstyle=\color{red},
            frame=b,         
 %        keywordstyle=[1]\textbf,    % Stil der Keywords
 %        keywordstyle=[2]\textbf,    %
 %        keywordstyle=[3]\textbf,    %
 %        keywordstyle=[4]\textbf,   \sqrt{\sqrt{}} %
         stringstyle=\color{white}\ttfamily, % Farbe der String
         showspaces=false,           % Leerzeichen anzeigen ?
         showtabs=false,             % Tabs anzeigen ?
         xleftmargin=17pt,
         framexleftmargin=17pt,
         framexrightmargin=5pt,
         framexbottommargin=4pt,
         %backgroundcolor=\color{lightgray},
         showstringspaces=false      % Leerzeichen in Strings anzeigen ?        
 }
 \lstloadlanguages{% Check Dokumentation for further languages ...
         %[Visual]Basic
         %Pascal
         %C
         %C++
         %XML
         %HTML
         Java
 }
    %\DeclareCaptionFont{blue}{\color{blue}} 

\definecolor{myblue}{RGB}{72, 165, 226}
\definecolor{myorange}{RGB}{222, 141, 8}
\titleformat{\paragraph}
{\normalfont\normalsize\bfseries}{\theparagraph}{1em}{}
\titlespacing*{\paragraph}
{0pt}{3.25ex plus 1ex minus .2ex}{1.5ex plus .2ex}
\setlength{\heavyrulewidth}{1.5pt}
\setlength{\abovetopsep}{4pt}
\setlength{\parindent}{0mm}
\linespread{1.3}
\DeclareMathOperator{\sech}{sech}
\DeclareMathOperator{\csch}{csch}
\DeclareMathOperator*{\argmin}{\arg\!\min}
\DeclareMathOperator{\Tr}{Tr}

\newcommand{\bs}{\boldsymbol}
\newcommand{\opn}{\operatorname}
%%%%%%%%%%%%%%%%%%%%%%%%%%%%%%%%%%%%%%%%%%%%%%%%%%%%%%%
% % We use newtheorem to define theorem-like structures
% %
% % Here are some common ones. . .
%%%%%%%%%%%%%%%%%%%%%%%%%%%%%%%%%%%%%%%%%%%%%%%%%%%%%%%
\newtheorem{theorem}{Theorem}
\newtheorem{lemma}{Lemma}
\newtheorem{proposition}{Proposition}
\newtheorem{scolium}{Scolium}   %% And a not so common one.
\newtheorem{definition}{Definition}
\newenvironment{proof}{{\sc Proof:}}{~\hfill QED}
\newenvironment{AMS}{}{}
\newenvironment{keywords}{}{}
%%%%%%%%%%%%%%%%%%%%%%%%%%%%%%%%%%%%%%%%%%%%%%%%%%%%%%%
% %   The first thanks indicates your affiliation
% %
% %  Just the name here.
% %
% % Your mailing address goes at the end.
% %
% % \thanks is also how you indicate grant support
% %
%%%%%%%%%%%%%%%%%%%%%%%%%%%%%%%%%%%%%%%%%%%%%%%%%%%%%%%


\begin{document}
\setstretch{1.5} %줄간격 조정
\newpage
\firstpageheader{}{}{\bf\large Daeyoung Lim \\ Grad School \\ Year of 2010, late}
\runningheader{Daeyoung Lim}{Graduate School Pre-exam}{2010 late}
\begin{questions}
   \question
    다음과 같은 다중선형회귀모형
    $$
      Y=\beta_{0}+\beta_{1}X_{1}+\cdots+\beta_{p}X_{p}+\epsilon
    $$
    을 고려하자. 여기서, $Y$는 독립변수를, $X_{1},\ldots,X_{p}$는 설명변수들을, 그리고 $\beta_{0},\beta_{1},\ldots,\beta_{p}$는 회귀계수들을 의미하며 $\epsilon$은 평균이 0, 분산이 $\sigma^{2}$인 오차항을 의미한다.
    \begin{enumerate}[(a)]
      \item 이 선형회귀모형을 적합한 후 얻어진 예측값을 $\widehat{Y}$으로 나타낼 때, $\widehat{Y}$의 기댓값 및 분산을 구하시오.
      \item 이 회귀모형의 잔차(residual)를 $e=Y-\widehat{Y}$으로 나타낼 때, $e$의 기댓값 및 분산을 구하시오.
      \item 잔차제곱합(residual sum of squares)의 기댓값을 구하시오.
    \end{enumerate}
    \begin{solution}
      \begin{enumerate}[(a)]
        \item 행렬꼴로 바꾸면 쉽다.
        \begin{align}
          \mathrm{E}\left(\widehat{Y}\right) &= \mathrm{E}\left(\mathbf{X}\widehat{\beta}\right)\\
          &=\mathbf{X}\beta\\
          \mathrm{Var}\left(\widehat{Y}\right) &= \mathrm{Var}\left(\mathbf{X}\widehat{\beta}\right)\\
          &= \mathbf{X}\mathrm{Var}\left(\widehat{\beta}\right)\mathbf{X}'\\
          &=\sigma^{2}\mathbf{X}\left(\mathbf{X}'\mathbf{X}\right)^{-1}\mathbf{X}'
        \end{align}
        \item 이미 구해놨으니 더 쉽다.
        \begin{align}
          \mathrm{E}\left(Y-\widehat{Y}\right) &= \mathbf{X}\beta-\mathbf{X}\beta\\
          &= 0\\
          \mathrm{Var}\left(Y-\widehat{Y}\right) &= \mathrm{Var}\left(\left(\mathbf{I}-\mathbf{H}\right)Y\right)\\
          &= \left(\mathbf{I}-\mathbf{H}\right)\sigma^{2}\mathbf{I}\left(\mathbf{I}-\mathbf{H}\right)'\\
          &=\sigma^{2}\left(\mathbf{I}-\mathbf{H}\right)
        \end{align}
        \item 잔차제곱합의 기댓값은 다음과 같다.
        \begin{align}
          \left(Y-\mathbf{H}Y\right)'\left(Y-\mathbf{H}Y\right) &= Y'\left(\mathbf{I}-\mathbf{H}\right)Y\\
          &= \left(Y-\mathbf{X}\beta\right)'\left(\mathbf{I}-\mathbf{H}\right)\left(Y-\mathbf{X}\beta\right)\\
          &= \epsilon'\left(\mathbf{I}-\mathbf{H}\right)\epsilon\\
          \mathrm{E}\left(\epsilon'\left(\mathbf{I}-\mathbf{H}\right)\epsilon\right) &= \Tr\left(\left(\mathbf{I}-\mathbf{H}\right)\mathrm{E}\left(\epsilon\epsilon'\right)\right)\\
          &=\sigma^{2}\left(\Tr\left(\mathbf{I}\right)-\Tr\left(\mathbf{H}\right)\right)\\
          &= \sigma^{2}\left(n-p-1\right)
        \end{align}
        여기서 중요한 것은 (11)에서 (12)로 넘어갈 수 있느냐 없느냐이다. (12)를 전개하면 (11)을 제외한 나머지 항들은 모두 소거되어 없어진다.
      \end{enumerate}
    \end{solution}
    \question
    Suppose that the conditional distribution of $X$ given that $P=p$ has a binomial distribution wit parameters 5 and $p$, $X\;|\;P=p \sim \mathrm{Bin}\left(5,p\right)$ and the marginal distribution of $P$ is a uniform distribution on $(0,1)$, $P\sim \mathrm{Unif}\left(0,1\right)$. We would like to calculate the correlation coefficient between $X$ and $P$.
    \begin{enumerate}[(a)]
      \item Compute variance of $X$.
      \item Compute covariance of $X$ and $P$.
      \item Compute $\mathrm{Cor}\left(X,P\right)$.
    \end{enumerate}
    \begin{solution}

    \end{solution}
    \question
    Let $X_{1},X_{2},\ldots,X_{n}\;\;\left(n>2\right)$ be a random sample from the following density
    $$
      f\left(x;\theta\right)=\begin{cases}\theta x^{\theta-1}, & 0<x<1,\;0<\theta<\infty \\ 0, & \text{elsewhere}\end{cases}
    $$
    \begin{enumerate}[(a)]
      \item Find the maximum likelhood estimator (MLE) $\widehat{\theta}$ of $\theta$.
      \item Compare variacne of $\widehat{\theta}$ with the Cramér-Rao lower bound.
    \end{enumerate}
    \begin{solution}

    \end{solution}
    \question
    Let $X_{1},\ldots,X_{n}$ be a random sample from the following probability density function(pdf),
    $$
      f\left(x;\theta\right)=\theta/x^{2},\quad 0<\theta\leq x<\infty.
    $$
    \begin{enumerate}
      \item Find a sufficient statistic for $\theta$.
      \item Find the maximum likelihood estimator (MLE) of $\theta$.
      \item Find the method of moments estimator (MME) of $\theta$.
    \end{enumerate}
    \begin{solution}

    \end{solution}
    \question
    Let $X_{1},\ldots, X_{n}$ be a random sample from the following probability density function (pdf),
    $$
      f\left(x;\theta\right)=\theta e^{-\theta x}, \quad 0<x<\infty,
    $$
    where $\theta=\theta_{0}$ or $\theta=\theta_{1}$. We assume that known fixed numbers $\theta_{1}>\theta_{0}$.
    \begin{enumerate}[(a)]
      \item Explain the \emph{Neyman-Pearson lemma} briefly.
      \item Explain the most powerful (MP) test briefly.
      \item Obtain the MP test for testing $H_{0}:\theta=\theta_{0}$ versus $H_{1}:\theta=\theta_{1}$ by the \emph{Neyman-Pearson lemma}.
    \end{enumerate}
    \begin{solution}

    \end{solution}
\end{questions}
\end{document}
