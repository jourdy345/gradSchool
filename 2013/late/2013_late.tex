\title{Graduate School Pre-exam Solution}
\author{Daeyoung Lim}

\documentclass[answers]{exam}
\usepackage[left=3cm,right=3cm,top=3.5cm,bottom=2cm]{geometry}
\usepackage{amssymb,amsmath}
\usepackage{mathtools}
\usepackage{graphicx}
\usepackage{kotex}
\usepackage[utf8]{inputenc}
\usepackage[T1]{fontenc}
\usepackage{lmodern}
% \usepackage{enumerate}
\usepackage{listings}
\usepackage{courier}
\usepackage{cancel}
\usepackage{array}
\usepackage{courier}
\usepackage{booktabs}
\usepackage{titlesec}
\usepackage[shortlabels]{enumitem}
\usepackage{setspace}
\usepackage{empheq}
\usepackage{tikz}
\usepackage{listings}

% \usepackage[toc,page]{appendix}

\setlength{\heavyrulewidth}{1.5pt}
\setlength{\abovetopsep}{4pt}

\DeclarePairedDelimiter{\ceil}{\lceil}{\rceil}
\newcommand\encircle[1]{%
  \tikz[baseline=(X.base)] 
    \node (X) [draw, shape=circle, inner sep=0] {\strut #1};}
 
% Command "alignedbox{}{}" for a box within an align environment
% Source: http://www.latex-community.org/forum/viewtopic.php?f=46&t=8144
\newlength\dlf  % Define a new measure, dlf
\newcommand\alignedbox[2]{
% Argument #1 = before & if there were no box (lhs)
% Argument #2 = after & if there were no box (rhs)
&  % Alignment sign of the line
{
\settowidth\dlf{$\displaystyle #1$}  
    % The width of \dlf is the width of the lhs, with a displaystyle font
\addtolength\dlf{\fboxsep+\fboxrule}  
    % Add to it the distance to the box, and the width of the line of the box     ㅊ
\hspace{-\dlf}  
    % Move everything dlf units to the left, so that & #1 #2 is aligned under #1 & #2
\boxed{#1 #2}
    % Put a box around lhs and rhs
}
}
\setcounter{secnumdepth}{4}
\lstset{
         basicstyle=\footnotesize\ttfamily, % Standardschrift
         %numbers=left,               % Ort der Zeilennummern
         numberstyle=\tiny,          % Stil der Zeilennummern
         %stepnumber=2,               % Abstand zwischen den Zeilennummern
         numbersep=5pt,              % Abstand der Nummern zum Text
         tabsize=2,                  % Groesse von Tabs
         extendedchars=true,         %
         breaklines=true,            % Zeilen werden Umgebrochen
         keywordstyle=\color{red},
            frame=b,         
 %        keywordstyle=[1]\textbf,    % Stil der Keywords
 %        keywordstyle=[2]\textbf,    %
 %        keywordstyle=[3]\textbf,    %
 %        keywordstyle=[4]\textbf,   \sqrt{\sqrt{}} %
         stringstyle=\color{white}\ttfamily, % Farbe der String
         showspaces=false,           % Leerzeichen anzeigen ?
         showtabs=false,             % Tabs anzeigen ?
         xleftmargin=17pt,
         framexleftmargin=17pt,
         framexrightmargin=5pt,
         framexbottommargin=4pt,
         %backgroundcolor=\color{lightgray},
         showstringspaces=false      % Leerzeichen in Strings anzeigen ?        
 }
 \lstloadlanguages{% Check Dokumentation for further languages ...
         %[Visual]Basic
         %Pascal
         %C
         %C++
         %XML
         %HTML
         Java
 }
    %\DeclareCaptionFont{blue}{\color{blue}} 

\definecolor{myblue}{RGB}{72, 165, 226}
\definecolor{myorange}{RGB}{222, 141, 8}
\titleformat{\paragraph}
{\normalfont\normalsize\bfseries}{\theparagraph}{1em}{}
\titlespacing*{\paragraph}
{0pt}{3.25ex plus 1ex minus .2ex}{1.5ex plus .2ex}
\setlength{\heavyrulewidth}{1.5pt}
\setlength{\abovetopsep}{4pt}
\setlength{\parindent}{0mm}
\linespread{1.3}
\DeclareMathOperator{\sech}{sech}
\DeclareMathOperator{\csch}{csch}
\DeclareMathOperator*{\argmin}{\arg\!\min}
\DeclareMathOperator{\Tr}{Tr}

\newcommand{\bs}{\boldsymbol}
\newcommand{\opn}{\operatorname}
%%%%%%%%%%%%%%%%%%%%%%%%%%%%%%%%%%%%%%%%%%%%%%%%%%%%%%%
% % We use newtheorem to define theorem-like structures
% %
% % Here are some common ones. . .
%%%%%%%%%%%%%%%%%%%%%%%%%%%%%%%%%%%%%%%%%%%%%%%%%%%%%%%
\newtheorem{theorem}{Theorem}
\newtheorem{lemma}{Lemma}
\newtheorem{proposition}{Proposition}
\newtheorem{scolium}{Scolium}   %% And a not so common one.
\newtheorem{definition}{Definition}
\newenvironment{proof}{{\sc Proof:}}{~\hfill QED}
\newenvironment{AMS}{}{}
\newenvironment{keywords}{}{}
%%%%%%%%%%%%%%%%%%%%%%%%%%%%%%%%%%%%%%%%%%%%%%%%%%%%%%%
% %   The first thanks indicates your affiliation
% %
% %  Just the name here.
% %
% % Your mailing address goes at the end.
% %
% % \thanks is also how you indicate grant support
% %
%%%%%%%%%%%%%%%%%%%%%%%%%%%%%%%%%%%%%%%%%%%%%%%%%%%%%%%


\begin{document}
\setstretch{1.5} %줄간격 조정
\newpage
\firstpageheader{}{}{\bf\large Daeyoung Lim \\ Grad School \\ Year of 2013, late}
\runningheader{Daeyoung Lim}{Graduate School Pre-exam}{2013 late}
\begin{questions}
   \question
    (25점) Let $X$ and $Y$ be independent non-negative random variables with continuous density functions $f_{X}\left(x\right)$ and $f_{Y}\left(y\right)$ respectively on $\left(0,\infty\right)$. Show that
    \begin{enumerate}[(a)]
      \item If, given $X+Y=u$, $X$ is uniformly distributed on $(0,u)$ whatever the value of $u$, then
      $$
        f_{Y}\left(u-v\right)f_{X}\left(v\right) = \dfrac{1}{u}\int_{0}^{u}f_{Y}\left(u-y\right)f_{X}\left(y\right)\,dy.
      $$
      \item If $X$ and $Y$ be independent exponential random variables with a common parameter $\lambda$, $X,Y\sim \mathrm{Exp}\left(\lambda\right)$, then the conditional distribution of $X$ given $X+Y=u$ is a uniform distribution on $\left(0,u\right)$.
    \end{enumerate}
    \begin{solution}
      \begin{enumerate}[(a)]
        \item 처음 보고 풀기 조금 어려운 문제인 듯하다. 좌변을 먼저 보면, $X+Y=U$와 $X=V$의 결합분포를 구한 것이다. 즉,
        \begin{align}
          \begin{cases}U=X+Y\\V=X  \end{cases} &\implies \begin{cases}X=V\\Y=U-V \end{cases}\\
          \left|J\right| &= 1\\
          f_{U,V}\left(u,v\right) &= f_{X,Y}\left(v,u-v\right)\cdot 1\\
          &= f_{Y}\left(u-v\right)f_{X}\left(v\right)
        \end{align}
        그리고 우변은 $U,V$의 분포를
        \begin{equation}
          f_{V|U}\left(v\,|\,u\right)f_{U}\left(u\right)
        \end{equation}
        의 꼴로 구한 것이다. 문제에서 $V\,|\,U\sim\mathrm{Unif}\left(0,u\right)$라고 했고 
        \begin{align}
        f_{U}\left(u\right) &= \int_{0}^{u}f_{U,V}\left(u,v\right)\,dv\\
        &= \int_{0}^{u}f_{Y}\left(u-v\right)f_{X}\left(v\right)\,dv
        \end{align}
        이므로
        \begin{equation}
          f_{V|U}\left(v|u\right)f_{U}\left(u\right)= \dfrac{1}{u}\int_{0}^{u}f_{Y}\left(u-v\right)f_{X}\left(v\right)\,dv
        \end{equation}
        이다. 둘 모두 $U,V$의 결합분포이므로 문제에서 주어진 등식이 성립한다.
        \item (a)에서 한 것을 바탕으로 계산하면
        \begin{align}
          f_{U,V}\left(u,v\right) &= f_{Y}\left(u-v\right)f_{X}\left(v\right)\\
          &= \lambda e^{-\lambda\left(u-v\right)}\cdot \lambda e^{-\lambda v}\\
          &=\lambda^{2}e^{-\lambda u}
        \end{align}
        따라서 $V\,|\,U$는 $U,V$의 결합분포에서 $U$의 주변분포를 나눠야 하므로
        \begin{align}
          f_{V|U}\left(v|u\right) &= \left.f_{U,V}\left(u,v\right)\middle/\int_{0}^{u}f_{U,V}\left(u,v\right) \,dv\right.\\
          &=\dfrac{1}{u}
        \end{align}
        그러므로 $V\,|\,U\sim \mathrm{Unif}\left(0,u\right)$이다.
      \end{enumerate}
    \end{solution}
    \question
    (25점) $X_{1},X_{2},\ldots,X_{n}$을 $\mathrm{Unif}\left(0,\theta\right)$로부터 얻은 랜덤표본이라고 하자.
    \begin{enumerate}[(a)]
      \item 모수 $\theta$의 최대가능도 추정량을 구하라.
      \item 위의 (a)에서 구한 모수 $\theta$의 최대가능도 추정량이 완비충분통계량임을 보여라.
      \item 모수 $\theta$에 대한 $\left(1-\alpha_{1}-\alpha_{2}\right)\times 100\%$ 신뢰구간을 구하라.
    \end{enumerate}
    \begin{solution}
      \begin{enumerate}[(a)]
        \item 이런 거 물어보지 마.
        \begin{equation}
          \widehat{\theta}=\max_{1\leq i \leq n}X_{i} \quad \left(=X_{\left(n\right)}\right)
        \end{equation}
        \item 완비통계량의 정의상 어떤 통계량 $T$가 있을 때, 임의의 $\theta\in \Omega$에 대해서
        \begin{equation}
          \mathrm{E}\left(r\left(T\right)\right)=0 \implies \mathrm{Pr}\left(r\left(T\right)=0\right)=1
        \end{equation}
        이어야 하므로 우리의 통계량 $X_{\left(n\right)}$의 분포로부터 어떤 함수꼴 $g\left(X_{\left(n\right)}\right)$의 기댓값과 $g\left(X_{\left(n\right)}\right)=0$일 확률 사이의 관계를 알아보면 된다.
        \begin{align}
          \mathrm{Pr}\left(X_{\left(n\right)}\leq x\right) &= \left(\mathrm{Pr}\left(X_{1}\leq x\right)\right)^{n}\\
          &=\left(\dfrac{x}{\theta}\right)^{n}\\
          f_{X_{\left(n\right)}}\left(x\right) &= nx^{n-1}\theta^{-n}
        \end{align}
        따라서
        \begin{align}
          \mathrm{E}\left(g\left(X_{\left(n\right)}\right)\right) &= \int_{0}^{\theta}nx^{n-1}g\left(x\right)\theta^{-n}\,dx\\
          &=n\theta^{-n}\int_{0}^{\theta}x^{n-1}g\left(x\right)\,dx=0
        \end{align}
        우리가 그러므로 알아봐야 할 관계는 다음과 같다.
        \begin{equation}
          \int_{0}^{\theta}x^{n-1}g\left(x\right)\,dx=0 \implies g\left(x\right)=0
        \end{equation}
        미적분의 기본정리에 의해
        \begin{align}
          \dfrac{d}{d\theta}\int_{0}^{\theta}x^{n-1}g\left(x\right)\,dx &= \theta^{n-1}g\left(\theta\right)\\
          &=0\\
          g\left(\theta\right) &= 0
        \end{align}
        0보다 큰 그 어떤 모수값 $\theta$를 가져와도 항상 $g\left(\theta\right)=0$이므로 $g:\mathbb{R}_{+}\mapsto \left\{0\right\}$이다.\par
        충분통계량임은 \emph{Neyman-Fisher factorization theorem}을 통해 밝힐 수 있다.
        \item 신뢰구간을 구하기 위해서는 주축통계량(pivot quantity)를 구해야 한다. 주축통계량으로 $X_{\left(n\right)}/\theta$를 생각할 수 있다. $Y=X_{\left(n\right)}/\theta$라 놓으면
        \begin{equation}
          f_{Y}\left(y\right) = ny^{n-1}
        \end{equation}
        이 되어 $Y\sim \mathrm{Be}\left(n,1\right)$이 됨을 알 수 있다. 모수가 $n$과 $1$인 베타분포의 CDF를 $F$라 할 때
        \begin{equation}
          \mathrm{Pr}\left(F_{\left(1-\alpha_{1}-\alpha_{2}\right)/2}^{-1}\leq \dfrac{X_{\left(n\right)}}{\theta}\leq F_{\left(1+\alpha_{1}+\alpha_{2}\right)/2}^{-1}\right)=1-\alpha_{1}-\alpha_{2}
        \end{equation}
        이므로 신뢰구간은
        \begin{equation}
          F_{\left(1+\alpha_{1}+\alpha_{2}\right)/2}^{-1}\cdot X_{\left(n\right)}\leq \theta\leq F_{\left(1-\alpha_{1}-\alpha_{2}\right)/2}^{-1}\cdot X_{\left(n\right)}
        \end{equation}
        가 될 것이다.
      \end{enumerate}
    \end{solution}
    \question
    (25점) $X_{1},\ldots,X_{n}$이 다음의 확률밀도함수를 가지는 랜덤표본일 때,
    $$
      f\left(x;\theta\right)=\theta x^{\theta-1},\quad 0\leq x \leq 1,\, \theta>0
    $$
    다음의 가설을 검정하고자 한다.
    $$
      H_{0}:\theta=3 \quad \text{vs} \quad H_{1}:\theta\neq 3
    $$
    이 가설에 대한 가능도비 검정을 $\theta$에 대한 최대가능도 추정량(MLE)의 함수로 표현하시오. (즉, MLE의 값에 따라 언제 귀무가설을 기각할 수 있는지 표현하시오.)
    \begin{solution}
      우선 $X_{i}\sim \mathrm{Be}\left(\theta,1\right)$이다. 2010년 후기 3번 문제에서 구했듯이
        \begin{align}
          \widehat{\theta}^{\text{MLE}} &= \left.n\middle/\left(-\sum_{i=1}^{n}\ln X_{i}\right)\right.\\
          &\sim \mathrm{InvGam}\left(n,n\theta\right)
        \end{align}
        이다. 그러므로 가능도비를 구하면
        \begin{align}
          \dfrac{L_{0}}{\widehat{L}} = \left(\dfrac{3}{\widehat{\theta}}\right)^{n}\left(\prod_{i=1}^{n}x_{i}\right)^{3-\widehat{\theta}} &< c_{1}\\
          n\left(\ln3-\ln\widehat{\theta}\right)+\left(3-\widehat{\theta}\right)\sum_{i=1}^{n}\ln x_{i}&<c_{2}\\
          -n\ln\widehat{\theta}-\dfrac{3n}{\widehat{\theta}}&<c_{3}\\
          \ln\widehat{\theta}+\dfrac{3}{\widehat{\theta}}&>c_{4}
        \end{align}
        도함수와 이계도함수를 구해서 최솟값이 어디인지, 그리고 변곡점이 어디인지 구하면 대충 그래프의 개형이 나온다. $f\left(x\right)=\ln x+3/x$는 $x>0$인 반직선 위에서 내려갔다 올라가므로 기각역은
        \begin{equation}
          \mathrm{RR}=\left\{\widehat{\theta}\,\middle|\,\ln\widehat{\theta}+\dfrac{3}{\widehat{\theta}}<a\quad \text{or}\quad \ln\widehat{\theta}+\dfrac{3}{\widehat{\theta}}>b \right\}
        \end{equation}
        이고 정확한 $a,b$값을 알려면 $\ln\widehat{\theta}+3/\widehat{\theta}$의 분포로 계산을 해야 하는데 변환 자체가 bijective하지 않아서 invertible하지 않으므로 수학적으로 상당히 복잡해진다. 따라서 
        \begin{equation}
          -2\ln\Lambda \xrightarrow{d}\chi_{L}^{2}
        \end{equation}
        라는 점근분포(asymptotic distribution)를 이용하는 편이 쉽다.
      \end{solution}
    \question
    (25점) 선형회귀모형 $Y=\beta_{0}+\beta_{1}X_{1}+\cdots+\beta_{p}X_{p}+\epsilon$을 고려하자. 여기서, $Y$는 독립변수를, $X_{1},\ldots,X_{p}$는 설명변수들을, 그리고 $\beta_{0},\ldots, \beta_{p}$는 회귀계수들을 의미하며 $\epsilon$은 평균이 $0$, 분산이 $\sigma^{2}$인 오차항을 의미한다.
    \begin{enumerate}[(a)]
      \item $p=1$인 단순선형회귀모형에서 결정계수(coefficient of determination) $R^{2}$는 $Y$와 $X_{1}$ 사이의 표본상관계수(correlation coefficient)의 제곱과 동일함을 보여라.
      \item 결정계수는 설명변수 $X_{1},\ldots,X_{p}$들의 측정단위에 의존하지 않음을 보여라.
    \end{enumerate}
    \begin{solution}
      \begin{enumerate}[(a)]
        \item \begin{align}
          R^{2} &= 1-\dfrac{\text{SSE}}{\text{SST}}\\
          &= 1-\dfrac{\sum_{i=1}^{n} \left(y_{i}-\hat{y}_{i}\right)^{2}}{\sum_{i=1}^{n}\left(y_{i}-\overline{y}\right)^{2}}\\
          &=1-\dfrac{\sum_{i=1}^{n} \left(y_{i}-\widehat{\beta}_{0}-\widehat{\beta}_{1}x_{1}\right)^{2}}{\sum_{i=1}^{n} \left(y_{i}-\overline{y}\right)^{2}}\\
          &=1-\dfrac{\sum_{i=1}^{n}\left(y_{i}-\overline{y}-\widehat{\beta}_{1}\left(x_{i}-\overline{x}\right)\right)^{2}}{\sum_{i=1}^{n}\left(y_{i}-\overline{y}\right)^{2}}\\
          &=1-\dfrac{\sum_{i=1}^{n}\left(y_{i}-\overline{y}\right)^{2}-2\widehat{\beta}_{1}\sum_{i=1}^{n}\left(x_{i}-\overline{x}\right)\left(y_{i}-\overline{y}\right)+\widehat{\beta}_{1}^{2}\sum_{i=1}^{n}\left(x_{i}-\overline{x}\right)^{2}}{\sum_{i=1}^{n}\left(y_{i}-\overline{y}\right)^{2}}\\
          &=\dfrac{\left(\sum_{i=1}^{n}\left(x_{i}-\overline{x}\right)\left(y_{i}-\overline{y}\right)\right)^{2}}{\sum_{i=1}^{n}\left(y_{i}-\overline{y}\right)^{2}\sum_{i=1}^{n}\left(x_{i}-\overline{x}\right)^{2}}\\
          &=\mathrm{Cor}^{2}\left(X,Y\right)
        \end{align}
        (40)에서 (41)로 넘어갈 때 $\widehat{\beta}_{1}$을 넣는다.
        \item 설계행렬을 $\mathbf{X}$라 하고 다음과 같이 표기한다.
        \begin{equation}
          \mathbf{X}= \begin{bmatrix}\mathbf{1}& \mathbf{c}_{1} & \cdots & \mathbf{c}_{p}  \end{bmatrix}
        \end{equation}
        여기서 $\mathbf{c}_{i}$는 $(i+1)$번째 열을 의미한다. 그렇다면 단위가 다른 설계행렬을 $\mathbf{X}_{s}$라 하고 다음과 같다고 하자.
        \begin{equation}
          \mathbf{X}_{s}= \begin{bmatrix}\mathbf{1}& d_{1}\mathbf{c}_{1} & \cdots & d_{p}\mathbf{c}_{p}  \end{bmatrix}
        \end{equation}
        여기서 $d_{i}$는 모두 상수이다. 이는 다시 다음과 같이 표기 할 수 있다.
        \begin{align}
          \mathbf{X}_{s}&= \mathbf{XD}\\
          \mathbf{D} &= \mathrm{diag}\left(1, d_{1},d_{2},\ldots ,d_{p}\right)
        \end{align}
        그리고 $Y$도 변환해서 $aY$로 놓는다. 이제 LSE를 다시 구해보면
        \begin{align}
          \widehat{\beta}_{s} &= a\mathbf{D}^{-1}\left(\mathbf{X}'\mathbf{X}\right)^{-1}\mathbf{X}'Y\\
          &=a\mathbf{D}^{-1}\widehat{\beta}
        \end{align}
        그리고 적합값(fitted values)도
        \begin{align}
          \widehat{Y}_{s} &= \mathbf{X}_{s}\widehat{\beta}_{s}\\
          &=a\mathbf{XDD}^{-1}\widehat{\beta}\\
          &=a\mathbf{X}\widehat{\beta}\\
          &=a\widehat{Y}
        \end{align}
        이제 $\mathrm{R}^{2}$를 다시 구해보면 다음과 같다. 원래가
        \begin{align}
          \mathrm{R}^{2} &= 1-\dfrac{\left(Y-\widehat{Y}\right)'\left(Y-\widehat{Y}\right)}{\left(Y-\overline{y}\mathbf{1}\right)'\left(Y-\overline{y}\mathbf{1}\right)}
        \end{align}
        라면
        \begin{align}
          \mathrm{R}_{s}^{2} &= 1-\dfrac{\left(aY-a\widehat{Y}\right)'\left(aY-a\widehat{Y}\right)}{\left(aY-a\overline{y}\mathbf{1}\right)'\left(aY-a\overline{y}\mathbf{1}\right)}\\
          &=\mathrm{R}^{2}
        \end{align}
      \end{enumerate}
    \end{solution}
\end{questions}
\end{document}
