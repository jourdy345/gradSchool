\title{Graduate School Pre-exam Solution}
\author{Daeyoung Lim}

\documentclass[answers]{exam}
\usepackage[left=3cm,right=3cm,top=3.5cm,bottom=2cm]{geometry}
\usepackage{amssymb,amsmath}
\usepackage{mathtools}
\usepackage{graphicx}
\usepackage{kotex}
\usepackage[utf8]{inputenc}
\usepackage[T1]{fontenc}
\usepackage{lmodern}
% \usepackage{enumerate}
\usepackage{listings}
\usepackage{courier}
\usepackage{cancel}
\usepackage{array}
\usepackage{courier}
\usepackage{booktabs}
\usepackage{titlesec}
\usepackage[shortlabels]{enumitem}
\usepackage{setspace}
\usepackage{empheq}
\usepackage{tikz}
\usepackage{listings}

% \usepackage[toc,page]{appendix}

\setlength{\heavyrulewidth}{1.5pt}
\setlength{\abovetopsep}{4pt}

\DeclarePairedDelimiter{\ceil}{\lceil}{\rceil}
\newcommand\encircle[1]{%
  \tikz[baseline=(X.base)] 
    \node (X) [draw, shape=circle, inner sep=0] {\strut #1};}
 
% Command "alignedbox{}{}" for a box within an align environment
% Source: http://www.latex-community.org/forum/viewtopic.php?f=46&t=8144
\newlength\dlf  % Define a new measure, dlf
\newcommand\alignedbox[2]{
% Argument #1 = before & if there were no box (lhs)
% Argument #2 = after & if there were no box (rhs)
&  % Alignment sign of the line
{
\settowidth\dlf{$\displaystyle #1$}  
    % The width of \dlf is the width of the lhs, with a displaystyle font
\addtolength\dlf{\fboxsep+\fboxrule}  
    % Add to it the distance to the box, and the width of the line of the box     ㅊ
\hspace{-\dlf}  
    % Move everything dlf units to the left, so that & #1 #2 is aligned under #1 & #2
\boxed{#1 #2}
    % Put a box around lhs and rhs
}
}
\setcounter{secnumdepth}{4}
\lstset{
         basicstyle=\footnotesize\ttfamily, % Standardschrift
         %numbers=left,               % Ort der Zeilennummern
         numberstyle=\tiny,          % Stil der Zeilennummern
         %stepnumber=2,               % Abstand zwischen den Zeilennummern
         numbersep=5pt,              % Abstand der Nummern zum Text
         tabsize=2,                  % Groesse von Tabs
         extendedchars=true,         %
         breaklines=true,            % Zeilen werden Umgebrochen
         keywordstyle=\color{red},
            frame=b,         
 %        keywordstyle=[1]\textbf,    % Stil der Keywords
 %        keywordstyle=[2]\textbf,    %
 %        keywordstyle=[3]\textbf,    %
 %        keywordstyle=[4]\textbf,   \sqrt{\sqrt{}} %
         stringstyle=\color{white}\ttfamily, % Farbe der String
         showspaces=false,           % Leerzeichen anzeigen ?
         showtabs=false,             % Tabs anzeigen ?
         xleftmargin=17pt,
         framexleftmargin=17pt,
         framexrightmargin=5pt,
         framexbottommargin=4pt,
         %backgroundcolor=\color{lightgray},
         showstringspaces=false      % Leerzeichen in Strings anzeigen ?        
 }
 \lstloadlanguages{% Check Dokumentation for further languages ...
         %[Visual]Basic
         %Pascal
         %C
         %C++
         %XML
         %HTML
         Java
 }
    %\DeclareCaptionFont{blue}{\color{blue}} 

\definecolor{myblue}{RGB}{72, 165, 226}
\definecolor{myorange}{RGB}{222, 141, 8}
\titleformat{\paragraph}
{\normalfont\normalsize\bfseries}{\theparagraph}{1em}{}
\titlespacing*{\paragraph}
{0pt}{3.25ex plus 1ex minus .2ex}{1.5ex plus .2ex}
\setlength{\heavyrulewidth}{1.5pt}
\setlength{\abovetopsep}{4pt}
\setlength{\parindent}{0mm}
\linespread{1.3}
\DeclareMathOperator{\sech}{sech}
\DeclareMathOperator{\csch}{csch}
\DeclareMathOperator*{\argmin}{\arg\!\min}
\DeclareMathOperator{\Tr}{Tr}

\newcommand{\bs}{\boldsymbol}
\newcommand{\opn}{\operatorname}
%%%%%%%%%%%%%%%%%%%%%%%%%%%%%%%%%%%%%%%%%%%%%%%%%%%%%%%
% % We use newtheorem to define theorem-like structures
% %
% % Here are some common ones. . .
%%%%%%%%%%%%%%%%%%%%%%%%%%%%%%%%%%%%%%%%%%%%%%%%%%%%%%%
\newtheorem{theorem}{Theorem}
\newtheorem{lemma}{Lemma}
\newtheorem{proposition}{Proposition}
\newtheorem{scolium}{Scolium}   %% And a not so common one.
\newtheorem{definition}{Definition}
\newenvironment{proof}{{\sc Proof:}}{~\hfill QED}
\newenvironment{AMS}{}{}
\newenvironment{keywords}{}{}
%%%%%%%%%%%%%%%%%%%%%%%%%%%%%%%%%%%%%%%%%%%%%%%%%%%%%%%
% %   The first thanks indicates your affiliation
% %
% %  Just the name here.
% %
% % Your mailing address goes at the end.
% %
% % \thanks is also how you indicate grant support
% %
%%%%%%%%%%%%%%%%%%%%%%%%%%%%%%%%%%%%%%%%%%%%%%%%%%%%%%%


\begin{document}
\setstretch{1.5} %줄간격 조정
\newpage
\firstpageheader{}{}{\bf\large Daeyoung Lim \\ Grad School \\ Year of 2013, late}
\runningheader{Daeyoung Lim}{Graduate School Pre-exam}{2013 late}
\begin{questions}
   \question
    (25점) let $X$ and $Y$ be independent non-negative random variables with continuous density functions $f_{X}\left(x\right)$ and $f_{Y}\left(y\right)$ respectively on $\left(0,\infty\right)$. Show that
    \begin{enumerate}[(a)]
      \item If, given $X+Y=u$, $X$ is uniformly distributed on $(0,u)$ whatever the value of $u$, then
      $$
        f_{Y}\left(u-v\right)f_{X}\left(v\right) = \dfrac{1}{v}\int_{0}^{u}f_{Y}\left(u-y\right)f_{X}\left(y\right)\,dy.
      $$
      \item If $X$ and $Y$ be independent exponential random variables with a common parameter $\lambda$, $X,Y\sim \mathrm{Exp}\left(\lambda\right)$, then the conditional distribution of $X$ given $X+Y=u$ is a uniform distribution on $\left(0,u\right)$.
    \end{enumerate}
    \begin{solution}

    \end{solution}
    \question
    (25점) $X_{1},X_{2},\ldots,X_{n}$을 $\mathrm{Unif}\left(0,\theta\right)$로부터 얻은 랜덤표본이라고 하자.
    \begin{enumerate}[(a)]
      \item 모수 $\theta$의 최대가능도 추정량을 구하라.
      \item 위의 (a)에서 구한 모수 $\theta$의 최대가능도 추정량이 완비충분통계량임을 보여라.
      \item 모수 $\theta$에 대한 $\left(1-\alpha_{1}-\alpha_{2}\right)\times 100\%$ 신뢰구간을 구하라.
    \end{enumerate}
    \begin{solution}

    \end{solution}
    \question
    (25점) $X_{1},\ldots,X_{n}$이 다음의 확률밀도함수를 가지는 랜덤표본일 때,
    $$
      f\left(x;\theta\right)=\theta x^{\theta-1},\quad 0\leq x \leq 1,\, \theta>0
    $$
    다음의 가설을 검정하고자 한다.
    $$
      H_{0}:\theta=3 \quad \text{vs} \quad H_{1}:\theta\neq 3
    $$
    이 가설에 대한 가능도비 검정을 $\theta$에 대한 최대가능도 추정량(MLE)의 함수로 표현하시오. (즉, MLE의 값에 따라 언제 귀무가설을 기각할 수 있는지 표현하시오.)
    \begin{solution}

    \end{solution}
    \question
    (25점) 선형회귀모형 $Y=\beta_{0}+\beta_{1}X_{1}+\cdots+\beta_{p}X_{p}+\epsilon$을 고려하자. 여기서, $Y$는 독립변수를, $X_{1},\ldots,X_{p}$는 설명변수들을, 그리고 $\beta_{0},\ldots, \beta_{p}$는 회귀계수들을 의미하며 $\epsilon$은 평균이 $0$, 분산이 $\sigma^{2}$인 오차항을 의미한다.
    \begin{enumerate}[(a)]
      \item $p=1$인 단순선형회귀모형에서 결정계수(coefficient of determination) $R^{2}$는 $Y$와 $X_{1}$ 사이의 표본상관계수(correlation coefficient)의 제곱과 동일함을 보여라.
      \item 결정계수는 설명변수 $X_{1},\ldots,X_{p}$들의 측정단위에 의존하지 않음을 보여라.
    \end{enumerate}
    \begin{solution}

    \end{solution}
\end{questions}
\end{document}
