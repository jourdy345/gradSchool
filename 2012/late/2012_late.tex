\title{Graduate School Pre-exam Solution}
\author{Daeyoung Lim}

\documentclass[answers]{exam}
\usepackage[left=3cm,right=3cm,top=3.5cm,bottom=2cm]{geometry}
\usepackage{amssymb,amsmath}
\usepackage{graphicx}
\usepackage{kotex}
\usepackage[utf8]{inputenc}
\usepackage[T1]{fontenc}
\usepackage{lmodern}
% \usepackage{enumerate}
\usepackage{listings}
\usepackage{courier}
\usepackage{cancel}
\usepackage{array}
\usepackage{courier}
\usepackage{booktabs}
\usepackage{titlesec}
\usepackage[shortlabels]{enumitem}
\usepackage{setspace}
\usepackage{empheq}
\usepackage{tikz}
\usepackage{listings}

% \usepackage[toc,page]{appendix}

\setlength{\heavyrulewidth}{1.5pt}
\setlength{\abovetopsep}{4pt}


\newcommand\encircle[1]{%
  \tikz[baseline=(X.base)] 
    \node (X) [draw, shape=circle, inner sep=0] {\strut #1};}
 
% Command "alignedbox{}{}" for a box within an align environment
% Source: http://www.latex-community.org/forum/viewtopic.php?f=46&t=8144
\newlength\dlf  % Define a new measure, dlf
\newcommand\alignedbox[2]{
% Argument #1 = before & if there were no box (lhs)
% Argument #2 = after & if there were no box (rhs)
&  % Alignment sign of the line
{
\settowidth\dlf{$\displaystyle #1$}  
    % The width of \dlf is the width of the lhs, with a displaystyle font
\addtolength\dlf{\fboxsep+\fboxrule}  
    % Add to it the distance to the box, and the width of the line of the box     ㅊ
\hspace{-\dlf}  
    % Move everything dlf units to the left, so that & #1 #2 is aligned under #1 & #2
\boxed{#1 #2}
    % Put a box around lhs and rhs
}
}
\setcounter{secnumdepth}{4}
\lstset{
         basicstyle=\footnotesize\ttfamily, % Standardschrift
         %numbers=left,               % Ort der Zeilennummern
         numberstyle=\tiny,          % Stil der Zeilennummern
         %stepnumber=2,               % Abstand zwischen den Zeilennummern
         numbersep=5pt,              % Abstand der Nummern zum Text
         tabsize=2,                  % Groesse von Tabs
         extendedchars=true,         %
         breaklines=true,            % Zeilen werden Umgebrochen
         keywordstyle=\color{red},
            frame=b,         
 %        keywordstyle=[1]\textbf,    % Stil der Keywords
 %        keywordstyle=[2]\textbf,    %
 %        keywordstyle=[3]\textbf,    %
 %        keywordstyle=[4]\textbf,   \sqrt{\sqrt{}} %
         stringstyle=\color{white}\ttfamily, % Farbe der String
         showspaces=false,           % Leerzeichen anzeigen ?
         showtabs=false,             % Tabs anzeigen ?
         xleftmargin=17pt,
         framexleftmargin=17pt,
         framexrightmargin=5pt,
         framexbottommargin=4pt,
         %backgroundcolor=\color{lightgray},
         showstringspaces=false      % Leerzeichen in Strings anzeigen ?        
 }
 \lstloadlanguages{% Check Dokumentation for further languages ...
         %[Visual]Basic
         %Pascal
         %C
         %C++
         %XML
         %HTML
         Java
 }
    %\DeclareCaptionFont{blue}{\color{blue}} 

\definecolor{myblue}{RGB}{72, 165, 226}
\definecolor{myorange}{RGB}{222, 141, 8}
\titleformat{\paragraph}
{\normalfont\normalsize\bfseries}{\theparagraph}{1em}{}
\titlespacing*{\paragraph}
{0pt}{3.25ex plus 1ex minus .2ex}{1.5ex plus .2ex}
\setlength{\heavyrulewidth}{1.5pt}
\setlength{\abovetopsep}{4pt}
\setlength{\parindent}{0mm}
\linespread{1.3}
\DeclareMathOperator{\sech}{sech}
\DeclareMathOperator{\csch}{csch}
\DeclareMathOperator*{\argmin}{\arg\!\min}
\DeclareMathOperator{\Tr}{Tr}

\newcommand{\bs}{\boldsymbol}
\newcommand{\opn}{\operatorname}
%%%%%%%%%%%%%%%%%%%%%%%%%%%%%%%%%%%%%%%%%%%%%%%%%%%%%%%
% % We use newtheorem to define theorem-like structures
% %
% % Here are some common ones. . .
%%%%%%%%%%%%%%%%%%%%%%%%%%%%%%%%%%%%%%%%%%%%%%%%%%%%%%%
\newtheorem{theorem}{Theorem}
\newtheorem{lemma}{Lemma}
\newtheorem{proposition}{Proposition}
\newtheorem{scolium}{Scolium}   %% And a not so common one.
\newtheorem{definition}{Definition}
\newenvironment{proof}{{\sc Proof:}}{~\hfill QED}
\newenvironment{AMS}{}{}
\newenvironment{keywords}{}{}
%%%%%%%%%%%%%%%%%%%%%%%%%%%%%%%%%%%%%%%%%%%%%%%%%%%%%%%
% %   The first thanks indicates your affiliation
% %
% %  Just the name here.
% %
% % Your mailing address goes at the end.
% %
% % \thanks is also how you indicate grant support
% %
%%%%%%%%%%%%%%%%%%%%%%%%%%%%%%%%%%%%%%%%%%%%%%%%%%%%%%%


\begin{document}
\setstretch{1.5} %줄간격 조정
\newpage
\firstpageheader{}{}{\bf\large Daeyoung Lim \\ Grad School \\ Year of 2012, late}
\runningheader{Daeyoung Lim}{Graduate School Pre-exam}{2012 late}
\begin{questions}
   \question
   If $X$ and $Y$ are independent exponential random variables with respective means $1/\lambda_{1}$ and $1/\lambda_{2}$.
   \begin{enumerate}[(a)]
    \item Compute the distribution of $Z=\min\left(X,Y\right)$.
    \item What is the conditional distribution of $Z$ given that $Z=X$?
    \item Consider the function $\lambda_{X}\left(t\right)$ defined as follows:
    \begin{equation}
        \exp\left[-\int_{0}^{x}\lambda_{X}\left(t\right)\,dt \right]=1-F_{X}\left(x\right),
    \end{equation}
    where $F_{X}\left(x\right)=\mathrm{Pr}\left(X\leq x\right)$. The function $\lambda_{X}\left(t\right)$ is called the failure rate function of $X$. Show that the failure rate function of $X$ is constant for $t>0$.
    \item Show that
    \begin{equation}
        \mathrm{Pr}\left(X<Y\,|\,Z=t\right)=\dfrac{\lambda_{X}\left(t\right)}{\lambda_{X}\left(t\right)+\lambda_{Y}\left(t\right)},
    \end{equation}
    where $\lambda_{X}\left(t\right)$ and $\lambda_{Y}\left(t\right)$ are the failure rate functions of $X$ and $Y$.
   \end{enumerate}
   \begin{solution}
    Rate parameter로 표기하면 $X\sim \mathrm{Exp}\left(\lambda_{1}\right),\; Y\sim\mathrm{Exp}\left(\lambda_{2}\right)$이다.
    \begin{enumerate}[(a)]
      \item CDF부터 시작하면
      \begin{align}
        \mathrm{Pr}\left(\min\left(X,Y\right)\leq z\right) &= 1-\mathrm{Pr}\left(X>z,Y>z\right)\\
        &=1-\mathrm{Pr}\left(X>z\right)\mathrm{Pr}\left(Y>z\right)\\
        &=1-\left(\int_{z}^{\infty}\lambda_{1}e^{-\lambda_{1}x}\,dx\,\int_{z}^{\infty}\lambda_{2}e^{-\lambda_{2}y}\,dy\right)\\
        &=1-\left(-1+e^{-\lambda_{1}z}-1+e^{-\lambda_{2}z}\right)\\
        &= 3-e^{-\lambda_{1}z}-e^{-\lambda_{2}z}
      \end{align}
    \end{enumerate}

   \end{solution}
   \question
   One observation is taken on a discrete random variable $Y$ with a probability mass function $f\left(y\,|\,\theta\right) = \mathrm{Pr}\left(Y=y\,|\,\theta\right)$, where $\theta\in\left\{-1,0,1\right\}$. Find the maximum likelihood estimate (MLE) of $\theta$.
   \begin{table}[!htbp]
    \centering
      \begin{tabular}{*4c}
        \toprule
        $Y$ & $\mathrm{Pr}\left(Y=y\,|\,\theta=-1\right)$ & $\mathrm{Pr}\left(Y=y\,|\,\theta=0\right)$ & $\mathrm{Pr}\left(Y=y\,|\,\theta=1\right)$\\
        \toprule
        1 & $1/3$ & $1/4$ & $0$ \\
        \midrule
        2 & $1/3$ & $1/4$ & $0$ \\
        \midrule
        3 & $0$ & $1/4$ & $1/2$ \\
        \midrule
        4 & $1/6$ & $1/4$ & $1/2$ \\
        \midrule
        5 & $1/6$ & $0$ & $1/4$\\ 
        \bottomrule
      \end{tabular}
    \end{table}
    \begin{solution}

    \end{solution}
    \question
    $X_{1},X_{2},\ldots,X_{100}$이 포아송 분포 $\mathrm{Poi}\left(\lambda\right)$로부터 추출한 랜덤 표본이라 하자. $H_{0}:\lambda=\lambda_{0}$ vs $H_{1}:\lambda=\lambda_{1}(>\lambda_{0})$에 대한 검정을 고려하자.
    \begin{enumerate}[(a)]
      \item 균일최강력검정법(uniformly most powerful test)의 기각 영역을 구하시오.
      \item (a)에서 구한 검정법의 검정력 함수(power function)를 구하시오.
      \item 대립가설이 $H_{1}:\lambda>\lambda_{0}$이라면 균일최강력검정법(uniformly most powerful test)이 존재하는지 보이시오.
    \end{enumerate}
    \begin{solution}

    \end{solution}
    \question
    단순선형회귀모형에 따르는 $n$개의 자료
    \begin{equation}
      y_{i} = \beta_{0}+\beta_{1}x_{i}+\epsilon_{i},\;\;i=1,\ldots,n
    \end{equation}
    들을 고려하자.
    \begin{enumerate}[(a)]
      \item 식 (1)을 행렬식
      \begin{equation}
        Y=\mathbf{X}\beta+\epsilon
      \end{equation}
      으로 표현하고자 한다. 이 경우 $Y,\mathbf{X},\beta$ 및 $\epsilon$을 행렬 또는 벡터로 표현하시오. (5점)
      \item 모수 $\beta$에 대한 최소제곱추정량(least squares estimator) $\widehat{\beta}$을 행렬대수를 이용하여 정의하고, $\widehat{\beta}$이 정규방정식
      \begin{equation}
        \mathbf{X}'\mathbf{X}\widehat{\beta}=\mathbf{X}'Y
      \end{equation}
      을 만족함을 보이시오. (5점)
      \item 행렬 $\mathbf{X}$가 full rank일 충분조건을 쓰고 이 경우 최소제곱추정량을 구체적으로 구하시오. (5점)
      \item 행렬 $\mathbf{X}$가 full rank라는 가정 하에서 기울기 모수 $\beta_{1}$에 대한 95\% 신뢰구간을 제시하되 신뢰수준이 95\%가 될 조건을 나열하시오. (5점)
      \item 위에서 제시한 신뢰구간의 신뢰수준이 95\%가 됨을 증명하시오. (5점)
    \end{enumerate}
    \begin{solution}

    \end{solution}
\end{questions}
\end{document}
