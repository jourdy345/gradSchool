\title{Graduate School Pre-exam Solution}
\author{Daeyoung Lim}

\documentclass[answers]{exam}
\usepackage[left=3cm,right=3cm,top=3.5cm,bottom=2cm]{geometry}
\usepackage{amssymb,amsmath}
\usepackage{mathtools}
\usepackage{graphicx}
\usepackage{kotex}
\usepackage[utf8]{inputenc}
\usepackage[T1]{fontenc}
\usepackage{lmodern}
% \usepackage{enumerate}
\usepackage{listings}
\usepackage{courier}
\usepackage{cancel}
\usepackage{array}
\usepackage{courier}
\usepackage{booktabs}
\usepackage{titlesec}
\usepackage[shortlabels]{enumitem}
\usepackage{setspace}
\usepackage{empheq}
\usepackage{tikz}
\usepackage{listings}

% \usepackage[toc,page]{appendix}

\setlength{\heavyrulewidth}{1.5pt}
\setlength{\abovetopsep}{4pt}

\DeclarePairedDelimiter{\ceil}{\lceil}{\rceil}
\newcommand\encircle[1]{%
  \tikz[baseline=(X.base)] 
    \node (X) [draw, shape=circle, inner sep=0] {\strut #1};}
 
% Command "alignedbox{}{}" for a box within an align environment
% Source: http://www.latex-community.org/forum/viewtopic.php?f=46&t=8144
\newlength\dlf  % Define a new measure, dlf
\newcommand\alignedbox[2]{
% Argument #1 = before & if there were no box (lhs)
% Argument #2 = after & if there were no box (rhs)
&  % Alignment sign of the line
{
\settowidth\dlf{$\displaystyle #1$}  
    % The width of \dlf is the width of the lhs, with a displaystyle font
\addtolength\dlf{\fboxsep+\fboxrule}  
    % Add to it the distance to the box, and the width of the line of the box     ㅊ
\hspace{-\dlf}  
    % Move everything dlf units to the left, so that & #1 #2 is aligned under #1 & #2
\boxed{#1 #2}
    % Put a box around lhs and rhs
}
}
\setcounter{secnumdepth}{4}
\lstset{
         basicstyle=\footnotesize\ttfamily, % Standardschrift
         %numbers=left,               % Ort der Zeilennummern
         numberstyle=\tiny,          % Stil der Zeilennummern
         %stepnumber=2,               % Abstand zwischen den Zeilennummern
         numbersep=5pt,              % Abstand der Nummern zum Text
         tabsize=2,                  % Groesse von Tabs
         extendedchars=true,         %
         breaklines=true,            % Zeilen werden Umgebrochen
         keywordstyle=\color{red},
            frame=b,         
 %        keywordstyle=[1]\textbf,    % Stil der Keywords
 %        keywordstyle=[2]\textbf,    %
 %        keywordstyle=[3]\textbf,    %
 %        keywordstyle=[4]\textbf,   \sqrt{\sqrt{}} %
         stringstyle=\color{white}\ttfamily, % Farbe der String
         showspaces=false,           % Leerzeichen anzeigen ?
         showtabs=false,             % Tabs anzeigen ?
         xleftmargin=17pt,
         framexleftmargin=17pt,
         framexrightmargin=5pt,
         framexbottommargin=4pt,
         %backgroundcolor=\color{lightgray},
         showstringspaces=false      % Leerzeichen in Strings anzeigen ?        
 }
 \lstloadlanguages{% Check Dokumentation for further languages ...
         %[Visual]Basic
         %Pascal
         %C
         %C++
         %XML
         %HTML
         Java
 }
    %\DeclareCaptionFont{blue}{\color{blue}} 

\definecolor{myblue}{RGB}{72, 165, 226}
\definecolor{myorange}{RGB}{222, 141, 8}
\titleformat{\paragraph}
{\normalfont\normalsize\bfseries}{\theparagraph}{1em}{}
\titlespacing*{\paragraph}
{0pt}{3.25ex plus 1ex minus .2ex}{1.5ex plus .2ex}
\setlength{\heavyrulewidth}{1.5pt}
\setlength{\abovetopsep}{4pt}
\setlength{\parindent}{0mm}
\linespread{1.3}
\DeclareMathOperator{\sech}{sech}
\DeclareMathOperator{\csch}{csch}
\DeclareMathOperator*{\argmin}{\arg\!\min}
\DeclareMathOperator{\Tr}{Tr}

\newcommand{\bs}{\boldsymbol}
\newcommand{\opn}{\operatorname}
%%%%%%%%%%%%%%%%%%%%%%%%%%%%%%%%%%%%%%%%%%%%%%%%%%%%%%%
% % We use newtheorem to define theorem-like structures
% %
% % Here are some common ones. . .
%%%%%%%%%%%%%%%%%%%%%%%%%%%%%%%%%%%%%%%%%%%%%%%%%%%%%%%
\newtheorem{theorem}{Theorem}
\newtheorem{lemma}{Lemma}
\newtheorem{proposition}{Proposition}
\newtheorem{scolium}{Scolium}   %% And a not so common one.
\newtheorem{definition}{Definition}
\newenvironment{proof}{{\sc Proof:}}{~\hfill QED}
\newenvironment{AMS}{}{}
\newenvironment{keywords}{}{}
%%%%%%%%%%%%%%%%%%%%%%%%%%%%%%%%%%%%%%%%%%%%%%%%%%%%%%%
% %   The first thanks indicates your affiliation
% %
% %  Just the name here.
% %
% % Your mailing address goes at the end.
% %
% % \thanks is also how you indicate grant support
% %
%%%%%%%%%%%%%%%%%%%%%%%%%%%%%%%%%%%%%%%%%%%%%%%%%%%%%%%


\begin{document}
\setstretch{1.5} %줄간격 조정
\newpage
\firstpageheader{}{}{\bf\large Daeyoung Lim \\ Grad School \\ Year of 2012, late}
\runningheader{Daeyoung Lim}{Graduate School Pre-exam}{2012 late}
\begin{questions}
   \question
   If $X$ and $Y$ are independent exponential random variables with respective means $1/\lambda_{1}$ and $1/\lambda_{2}$.
   \begin{enumerate}[(a)]
    \item Compute the distribution of $Z=\min\left(X,Y\right)$.
    \item What is the conditional distribution of $Z$ given that $Z=X$?
    \item Consider the function $\lambda_{X}\left(t\right)$ defined as follows:
    \begin{equation}
        \exp\left[-\int_{0}^{x}\lambda_{X}\left(t\right)\,dt \right]=1-F_{X}\left(x\right),
    \end{equation}
    where $F_{X}\left(x\right)=\mathrm{Pr}\left(X\leq x\right)$. The function $\lambda_{X}\left(t\right)$ is called the failure rate function of $X$. Show that the failure rate function of $X$ is constant for $t>0$.
    \item Show that
    \begin{equation}
        \mathrm{Pr}\left(X<Y\,|\,Z=t\right)=\dfrac{\lambda_{X}\left(t\right)}{\lambda_{X}\left(t\right)+\lambda_{Y}\left(t\right)},
    \end{equation}
    where $\lambda_{X}\left(t\right)$ and $\lambda_{Y}\left(t\right)$ are the failure rate functions of $X$ and $Y$.
   \end{enumerate}
   \begin{solution}
    Rate parameter로 표기하면 $X\sim \mathrm{Exp}\left(\lambda_{1}\right),\; Y\sim\mathrm{Exp}\left(\lambda_{2}\right)$이다.
    \begin{enumerate}[(a)]
      \item CDF부터 시작하면
      \begin{align}
        \mathrm{Pr}\left(\min\left(X,Y\right)\leq z\right) &= 1-\mathrm{Pr}\left(X>z,Y>z\right)\\
        &=1-\mathrm{Pr}\left(X>z\right)\mathrm{Pr}\left(Y>z\right)\\
        &=1-\left(\int_{z}^{\infty}\lambda_{1}e^{-\lambda_{1}x}\,dx\,\int_{z}^{\infty}\lambda_{2}e^{-\lambda_{2}y}\,dy\right)\\
        &=1-e^{-\lambda_{1}z}e^{-\lambda_{2}z} \\
        &= 1-e^{-\left(\lambda_{1}+\lambda_{2}\right)z}\\
        \dfrac{d}{dz}\mathrm{Pr}\left(\min\left(X,Y\right)\leq z\right) &= \left(\lambda_{1}+\lambda_{2}\right)e^{-\left(\lambda_{1}+\lambda_{2}\right)z}, \;\;z>0
      \end{align}
      그러므로 $Z\sim\mathrm{Exp}\left(\lambda_{1}+\lambda_{2}\right)$이다.
      \item 정의대로 계산하면 된다.
      \begin{align}
        \mathrm{Pr}\left(Z\leq z\,|\,Z=X\right)&= 1-\mathrm{Pr}\left(Z\geq z\,|\,Z=X\right)\\
        &=1-\dfrac{\mathrm{Pr}\left(Z\geq z, X<Y\right)}{\mathrm{Pr}\left(X<Y\right)}\\
        &=1-\dfrac{\mathrm{Pr}\left(Z\leq X<Y\right)}{\mathrm{Pr}\left(X<Y\right)}\\
        \mathrm{Pr}\left(X<Y\right) &= \int_{0}^{\infty}\int_{0}^{y}\lambda_{1}e^{-\lambda_{1}x}\lambda_{2}e^{-\lambda_{2}y}\,dx\,dy\\
        &= \dfrac{\lambda_{1}}{\lambda_{1}+\lambda_{2}}\\
        \mathrm{Pr}\left(Z\leq X<Y\right) &= \int_{z}^{\infty}\int_{z}^{y}\lambda_{1}e^{-\lambda_{1}x}\lambda_{2}e^{-\lambda_{2}y}\,dx\,dy\\
        &=\dfrac{\lambda_{1}}{\lambda_{1}+\lambda_{2}}e^{-\left(\lambda_{1}+\lambda_{2}\right)z}\\
        \therefore \mathrm{Pr}\left(Z\leq z\,|\,Z=X\right) &= 1-e^{-\left(\lambda_{1}+\lambda_{2}\right)z}\\
        Z\,|\,Z=X &\sim \mathrm{Exp}\left(\lambda_{1}+\lambda_{2}\right)
      \end{align}
      where $\mathrm{Exp}\left(\lambda_{1}+\lambda_{2}\right)$ denotes the exponential distribution whose mean is $\left(\lambda_{1}+\lambda_{2}\right)^{-1}$.
      \item 위에서 풀어봤으면 알겠지만 $X\sim\mathrm{Exp}\left(\lambda_{1}\right)$인 $X$의 CDF는 $1-e^{-\lambda_{1} x}$, 고로 $1-F_{X}\left(x\right)=e^{-\lambda_{1} x}$이다. 즉, 미적분의 기본정리에 의해
      \begin{equation}
      \dfrac{d}{dx}\int_{0}^{x}\lambda_{X}\left(t\right)=\lambda_{X}\left(x\right)=\lambda_{1}
      \end{equation}
      로 상수가 나오게 된다.
      \item 이것도 그냥 정의대로 풀면 된다.
      \begin{align}
        \mathrm{Pr}\left(X<Y\,|\,Z=t\right) &=\dfrac{\mathrm{Pr}\left(X<Y,Z=t\right)}{\mathrm{Pr}\left(Z=t\right)}\\
        &=\dfrac{\mathrm{Pr}\left(X<Y,X=t\right)}{\mathrm{Pr}\left(Z=t\right)}\\
        &=\dfrac{\mathrm{Pr}\left(X=t,t<Y\right)}{\mathrm{Pr}\left(Z=t\right)}\\
        &=\dfrac{\mathrm{Pr}\left(X=t\right)\mathrm{Pr}\left(t<Y\right)}{\mathrm{Pr}\left(Z=t\right)},\qquad \text{적분은 스스로...}\\
        &=\dfrac{\lambda_{1}}{\lambda_{1}+\lambda_{2}}\\
        &=\dfrac{\lambda_{X}\left(t\right)}{\lambda_{X}\left(t\right)+\lambda_{Y}\left(t\right)}
      \end{align}
    \end{enumerate}
   \end{solution}
   \question
   One observation is taken on a discrete random variable $Y$ with a probability mass function $f\left(y\,|\,\theta\right) = \mathrm{Pr}\left(Y=y\,|\,\theta\right)$, where $\theta\in\left\{-1,0,1\right\}$. Find the maximum likelihood estimate (MLE) of $\theta$.
   \begin{table}[!htbp]
    \centering
      \begin{tabular}{*4c}
        \toprule
        $Y$ & $\mathrm{Pr}\left(Y=y\,|\,\theta=-1\right)$ & $\mathrm{Pr}\left(Y=y\,|\,\theta=0\right)$ & $\mathrm{Pr}\left(Y=y\,|\,\theta=1\right)$\\
        \toprule
        1 & $1/3$ & $1/4$ & $0$ \\
        \midrule
        2 & $1/3$ & $1/4$ & $0$ \\
        \midrule
        3 & $0$ & $1/4$ & $1/4$ \\
        \midrule
        4 & $1/6$ & $1/4$ & $1/2$ \\
        \midrule
        5 & $1/6$ & $0$ & $1/4$\\ 
        \bottomrule
      \end{tabular}
    \end{table}
    \begin{solution}
      표까지 그려가면서 정성스럽게 쉬운 문제를 냈는데, 그냥 각 관측값마다 확률이 (정확하게는 \emph{unnormalized likelihood function}이) 가장 큰 모수값을 선택해주면 된다. 그러니까
      \begin{equation} 
        \widehat{\theta}^{\text{MLE}}=\begin{cases}-1,& \text{if $Y=1$ or $Y=2$}\\0\text{ or }1,& \text{if Y=3}\\ 1,& \text{if $Y=4$ or $Y=5$} \end{cases}
      \end{equation}
    \end{solution}
    \question
    $X_{1},X_{2},\ldots,X_{100}$이 포아송 분포 $\mathrm{Poi}\left(\lambda\right)$로부터 추출한 랜덤 표본이라 하자. $H_{0}:\lambda=\lambda_{0}$ vs $H_{1}:\lambda=\lambda_{1}(>\lambda_{0})$에 대한 검정을 고려하자.
    \begin{enumerate}[(a)]
      \item 균일최강력검정법(uniformly most powerful test)의 기각 영역을 구하시오.
      \item (a)에서 구한 검정법의 검정력 함수(power function)를 구하시오.
      \item 대립가설이 $H_{1}:\lambda>\lambda_{0}$이라면 균일최강력검정법(uniformly most powerful test)이 존재하는지 보이시오.
    \end{enumerate}
    \begin{solution}
      \begin{enumerate}[(a)]
        \item 가능도비를 구한다.
        \begin{align}
          \dfrac{L_{0}}{L_{1}}&=\dfrac{e^{-\lambda_{0}n}\lambda_{0}^{X_{1}+\cdots+X_{n}}}{e^{-\lambda_{1}n}\lambda_{1}^{X_{1}+\cdots+X_{n}}}\\
          &=e^{\left(\lambda_{1}-\lambda_{0}\right)n}\left(\dfrac{\lambda_{0}}{\lambda_{1}}\right)^{X_{1}+\cdots+X_{n}}\leq k
        \end{align}
        일 때 귀무가설이 기각되는 것이 균일최강력검정법이다. $\lambda_{1}>\lambda_{0}$이므로 $\lambda_{0}$의 $\lambda_{1}$에 대한 비는 $1$보다 작다. 따라서 기각영역은 
        \begin{equation}
          \sum_{i=1}^{n}X_{i}\geq c
        \end{equation}
        일 때이며 (본 문제에서는 $n=100$), 상수 $c$는 다음과 같이 구한다.
        \begin{equation}
          \mathrm{Pr}\left(\sum_{i=1}^{n}X_{i}\geq c\,\middle|\,H_{0}\right)=\alpha
        \end{equation}
        여기서 $X_{1}+\cdots+X_{n}\sim\mathrm{Poi}\left(n\lambda\right)$이므로 (29)는
        \begin{equation}
          \sum_{x=\ceil{c}}^{\infty}\dfrac{e^{-n\lambda_{0}}\left(n\lambda_{0}\right)^{x}}{x!}=\alpha
        \end{equation}
        가 된다. 따라서 저 식을 만족하는 실수 $c$를 찾아서 그것보다 모든 변수를 다 더한 값이 크면 귀무가설을 기각한다.
        \item 검정력 함수는 대립가설을 참으로 놓고 기각역의 확률을 구하는 것이므로
        \begin{equation}
          \beta\left(X_{1},\ldots,X_{n}\right) = \sum_{x=\ceil{c}}^{\infty}\dfrac{e^{-n\lambda_{1}}\left(n\lambda_{1}\right)^{x}}{x!}
        \end{equation}
        이며 어차피 $c$는 정수로 나올 것이므로 $\ceil{c}=c$가 될 것이다. 그리고 $n=100$이다.
        \item 결국 위에서 사용한 대립가설의 정보는 $\lambda_{1}>\lambda_{0}$뿐이므로 대립가설이 그에 따라 바뀌어도 결과는 변함이 없다. 고로 균일최강력검정은 존재하고 그 기각역 및 모든 결과는 위와 동일하다.
      \end{enumerate}
    \end{solution}
    \question
    단순선형회귀모형에 따르는 $n$개의 자료
    \begin{equation}
      y_{i} = \beta_{0}+\beta_{1}x_{i}+\epsilon_{i},\;\;i=1,\ldots,n
    \end{equation}
    들을 고려하자.
    \begin{enumerate}[(a)]
      \item 식 (1)을 행렬식
      \begin{equation}
        Y=\mathbf{X}\beta+\epsilon
      \end{equation}
      으로 표현하고자 한다. 이 경우 $Y,\mathbf{X},\beta$ 및 $\epsilon$을 행렬 또는 벡터로 표현하시오. (5점)
      \item 모수 $\beta$에 대한 최소제곱추정량(least squares estimator) $\widehat{\beta}$을 행렬대수를 이용하여 정의하고, $\widehat{\beta}$이 정규방정식
      \begin{equation}
        \mathbf{X}'\mathbf{X}\widehat{\beta}=\mathbf{X}'Y
      \end{equation}
      을 만족함을 보이시오. (5점)
      \item 행렬 $\mathbf{X}$가 full rank일 충분조건을 쓰고 이 경우 최소제곱추정량을 구체적으로 구하시오. (5점)
      \item 행렬 $\mathbf{X}$가 full rank라는 가정 하에서 기울기 모수 $\beta_{1}$에 대한 95\% 신뢰구간을 제시하되 신뢰수준이 95\%가 될 조건을 나열하시오. (5점)
      \item 위에서 제시한 신뢰구간의 신뢰수준이 95\%가 됨을 증명하시오. (5점)
    \end{enumerate}
    \begin{solution}
      \begin{enumerate}[(a)]
      \item 예년도 문제 참조.
      \item 이 문제는 두 가지 방법으로 풀 수 있다. 하나는 가장 편하게 벡터미분을 이용하는 것이다. 즉, 문제를 수치최적화 문제로 치환한 뒤 변수벡터에 대해 미분한 후 0으로 놓고 정리하는 식이다.
      \begin{align}
        \widehat{\beta} &= \argmin_{\beta}\left(Y-\mathbf{X}\beta\right)'\left(Y-\mathbf{X}\beta\right)\\
        \dfrac{d}{d\beta}\left(Y-\mathbf{X}\beta\right)'\left(Y-\mathbf{X}\beta\right) &= -2\mathbf{X}'Y+2\mathbf{X}'\mathbf{X}\beta = 0\\
        \left(\mathbf{X}'\mathbf{X}\right)\widehat{\beta} &= \mathbf{X}'Y
      \end{align}
      따라서 최소제곱추정량은 이를 반드시 만족함을 알 수 있다. 다만 이 식이 반드시 성립하는지는 증명해야 한다. 이는 다음과 같이 증명한다. 행렬의 열공간(column space)는 $\bs{\mathrm{col}}\left(\mathbf{X}\right)$로 영공간(null space)는 $\bs{\mathrm{null}}\left(\mathbf{X}\right)$로 표현하기로 한다. \par
      \begin{lemma}
        For any $m\times n$ matrix $A$, $\bs{\mathrm{null}}\left(A'A\right)=\bs{\mathrm{null}}\left(A\right)$.
      \end{lemma}
      \begin{proof}
        벡터 $x$가 $\bs{\mathrm{null}}\left(A\right)$에 있다고 하자. 즉, $Ax=0$. 여기서 $A'Ax=0$도 성립함은 쉽게 알 수 있다. 반대로 만약 $x\in\bs{\mathrm{null}}\left(A'A\right)$라면 $A'Ax=0$이다. 다시 여기에 $x'$를 곱하면 $x'A'Ax=0$는 변함이 없고 이는 다시 $\left(Ax\right)'\left(Ax\right)=0$이 되어 $Ax$벡터의 내적이 된다. 내적이 0이기 위해서는 $Ax=0$이어야 하므로 $x\in\bs{\mathrm{null}}\left(A\right)$. 따라서 $x\in\bs{\mathrm{null}}\left(A'A\right)\iff x\in\bs{\mathrm{null}}\left(A\right)$.
      \end{proof}
      \begin{lemma}
        For any $m\times n$ matrix $A$, $\bs{\mathrm{col}}\left(A'\right)=\bs{\mathrm{col}}\left(A'A\right)$.
      \end{lemma}
      \begin{proof}
        임의의 $x\in\mathbb{R}^{n}$ 벡터를 택하면 $A'Ax$는 $\bs{\mathrm{col}}\left(A'A\right)$에 속한다. 이는 다시 $A'\left(Ax\right)$이므로 $\bs{\mathrm{col}}\left(A'\right)$에도 속한다. 따라서 $\bs{\mathrm{col}}\left(A'A\right)\subset \bs{\mathrm{col}}\left(A'\right)$이다. 이제 열공간의 차원을 비교하면 되는데 우회적으로 앞서 \emph{lemma}로 증명했던 영공간을 통해 $\bs{\mathrm{col}}\left(A'A\right)=\bs{\mathrm{col}}\left(A'\right)$임을 보일 수 있다. 왜냐하면 $\bs{\mathrm{null}}\left(A'A\right)=\bs{\mathrm{null}}\left(A\right)$이고 $A'A$와 $A$모두 열의 개수가 $n$개로 동일하기 때문에 \emph{rank-nullity theorem}에 의해 열공간의 차원 역시 동일함을 알 수 있다. 다시 \emph{rank theorem}에 따라 $\bs{\mathrm{col}}\left(A'\right)$의 차원과 $\bs{\mathrm{col}}\left(A\right)$의 차원이 동일하므로 $\bs{\mathrm{col}}\left(A'A\right)$의 차원과 $\bs{\mathrm{col}}\left(A'\right)$의 차원이 같다. 따라서 $\bs{\mathrm{col}}\left(A'A\right)=\bs{\mathrm{col}}\left(A'\right)$이다.\par
      \end{proof}
      이제 다시 돌아와서 $\mathbf{X}'\mathbf{X}\widehat{\beta}=\mathbf{X}'Y$가 항상 성립하는가에 대한 질문에 답해보자. 분명 위에 따르면 $\mathbf{X}'Y\in\bs{\mathrm{col}}\left(\mathbf{X}'\right)$이며 이는 곧 $\bs{\mathrm{col}}\left(\mathbf{X}'\mathbf{X}\right)$이므로 언제나 성립함을 알 수 있다. 단, $\widehat{\beta}$가 유일하게 결정되기 위해서는 $\mathbf{X}'\mathbf{X}$의 역행렬이 존재해야 하고 (일반적으로 행보다 열이 많은 $\mathbf{X}$를 가정하므로) $\bs{\mathrm{rank}}\left(\mathbf{X}'\mathbf{X}\right)=\text{\# of independent columns}$이기 때문에 $\mathbf{X}$의 열이 모두 선형독립이어야 함을 알 수 있다.
      \item (b) 참조.
      \begin{equation}
        \widehat{\beta}=\left(\mathbf{X}'\mathbf{X}\right)^{-1}\mathbf{X}'Y
      \end{equation}
      \item $\widehat{\beta}\sim\mathcal{N}\left(\beta,\sigma^{2}\left(\mathbf{X}'\mathbf{X}\right)^{-1}\right)$이므로 $\mathbf{C}=\left(\mathbf{X}'\mathbf{X}\right)^{-1}$라 놓으면 $\mathrm{Var}\left(\widehat{\beta}_{1}\right)=\sigma^{2}\mathbf{C}_{22}$. 단순선형회귀의 경우
      \begin{align}
        \mathbf{X}'\mathbf{X} &= \begin{bmatrix}1 & 1 &\cdots & 1\\ x_{1} &x_{2} & \cdots & x_{n}\end{bmatrix}\begin{bmatrix}1& x_{1}\\ 1 & x_{2} \\ \vdots & \vdots \\ 1 & x_{n}\end{bmatrix}\\
        &= \begin{bmatrix}n & \sum_{i=1}^{n}x_{i}\\ \sum_{i=1}^{n}x_{i} & \sum_{i=1}^{n}x_{i}^{2}\end{bmatrix}\\
        \left(\mathbf{X}'\mathbf{X}\right)^{-1} &= \dfrac{1}{n\sum_{i=1}^{n}x_{i}^{2}-\left(n\overline{x}_{n}\right)^{2}}\begin{bmatrix}\sum_{i=1}^{n}x_{i}^{2}& -\sum_{i=1}^{n}x_{i}\\ -\sum_{i=1}^{n}x_{i} & n \end{bmatrix}
      \end{align}
      여기서
      \begin{equation}
        n\sum_{i=1}^{n}x_{i}^{2}-\left(n\overline{x}_{n}\right)^{2} = n\sum_{i=1}^{n}\left(x_{i}-\overline{x}_{n}\right)^{2}
      \end{equation}
      따라서 $\mathbf{C}_{22}=1\Big/ \left(\sum_{i=1}^{n}\left(x_{i}-\overline{x}_{n}\right)^{2} \right)$이므로 ($\sigma^{2}$는 모르므로 비편향추정량인 $S^{2}$로 대입한다.)
      \begin{align}
        \mathrm{Var}\left(\widehat{\beta}_{1}\right) &= \dfrac{\displaystyle \sum_{i=1}^{n}\left(y_{i}-\hat{y}_{i}\right)^{2}}{\left(n-2\right)\displaystyle \sum_{i=1}^{n}\left(x_{i}-\overline{x}_{n}\right)^{2}}
      \end{align}
      그러므로
      \begin{equation}
        \dfrac{\widehat{\beta}_{1}-\beta_{1}}{\sqrt{\mathrm{Var}\left(\widehat{\beta}_{1}\right)}}\sim t_{n-2}
      \end{equation}
      를 이용하여
      \begin{align}
        \mathrm{Pr}&\left(-t_{n-2}\left(0.025\right)\leq \dfrac{\displaystyle\widehat{\beta}_{1}-\beta_{1}}{\sqrt{\mathrm{Var}\left(\widehat{\beta}_{1}\right)}} \leq t_{n-2}\left(0.025\right)\right)=0.95\\
        \mathrm{CI}&:\;\widehat{\beta}_{1}-t_{n-2}\left(0.025\right)\sqrt{\mathrm{Var}\left(\widehat{\beta}\right)} \leq \beta_{1} \leq \widehat{\beta}_{1}+t_{n-2}\left(0.025\right)\sqrt{\mathrm{Var}\left(\widehat{\beta}\right)}
      \end{align}
      (43)을 (46)에 대입한다. 이게 성립하려면 처음에 세웠던 모형이 맞아야 한다. 즉, 오차항이 정규분포를 따라야 하고 모두 같은 분산을 가져야 한다. 또한 $\mathbf{X}$와 오차항은 독립이어야 한다.
      \item 뭘 증명하라는 건지 모르겠다. By construction, 신뢰수준은 95\%이다.
      \end{enumerate}
    \end{solution}
\end{questions}
\end{document}
