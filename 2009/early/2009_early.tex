\title{Graduate School Pre-exam Solution}
\author{Daeyoung Lim}

\documentclass[answers]{exam}
\usepackage[left=3cm,right=3cm,top=3.5cm,bottom=2cm]{geometry}
\usepackage{amssymb,amsmath}
\usepackage{kotex}
\usepackage[utf8]{inputenc}
\usepackage[T1]{fontenc}
\usepackage{lmodern}
\usepackage{enumerate}
\usepackage{listings}
\usepackage{courier}
\usepackage{cancel}
\usepackage{array}
\usepackage{courier}
\usepackage{booktabs}
\usepackage{titlesec}
\usepackage{enumitem}
\usepackage{setspace}
% \usepackage[toc,page]{appendix}
\usepackage[usenames, dvipsnames]{color}
\setcounter{secnumdepth}{4}
\lstset{
         basicstyle=\footnotesize\ttfamily, % Standardschrift
         %numbers=left,               % Ort der Zeilennummern
         numberstyle=\tiny,          % Stil der Zeilennummern
         %stepnumber=2,               % Abstand zwischen den Zeilennummern
         numbersep=5pt,              % Abstand der Nummern zum Text
         tabsize=2,                  % Groesse von Tabs
         extendedchars=true,         %
         breaklines=true,            % Zeilen werden Umgebrochen
         keywordstyle=\color{red},
            frame=b,         
 %        keywordstyle=[1]\textbf,    % Stil der Keywords
 %        keywordstyle=[2]\textbf,    %
 %        keywordstyle=[3]\textbf,    %
 %        keywordstyle=[4]\textbf,   \sqrt{\sqrt{}} %
         stringstyle=\color{white}\ttfamily, % Farbe der String
         showspaces=false,           % Leerzeichen anzeigen ?
         showtabs=false,             % Tabs anzeigen ?
         xleftmargin=17pt,
         framexleftmargin=17pt,
         framexrightmargin=5pt,
         framexbottommargin=4pt,
         %backgroundcolor=\color{lightgray},
         showstringspaces=false      % Leerzeichen in Strings anzeigen ?        
 }
 \lstloadlanguages{% Check Dokumentation for further languages ...
         %[Visual]Basic
         %Pascal
         %C
         %C++
         %XML
         %HTML
         Java
 }
    %\DeclareCaptionFont{blue}{\color{blue}} 

\definecolor{myblue}{RGB}{72, 165, 226}
\definecolor{myorange}{RGB}{222, 141, 8}
\titleformat{\paragraph}
{\normalfont\normalsize\bfseries}{\theparagraph}{1em}{}
\titlespacing*{\paragraph}
{0pt}{3.25ex plus 1ex minus .2ex}{1.5ex plus .2ex}
\setlength{\heavyrulewidth}{1.5pt}
\setlength{\abovetopsep}{4pt}
\setlength{\parindent}{0mm}
\linespread{1.3}
\DeclareMathOperator{\sech}{sech}
\DeclareMathOperator{\csch}{csch}
\DeclareMathOperator*{\argmin}{\arg\!\min}
\DeclareMathOperator{\Tr}{Tr}

\newcommand{\bs}{\boldsymbol}
\newcommand{\opn}{\operatorname}
%%%%%%%%%%%%%%%%%%%%%%%%%%%%%%%%%%%%%%%%%%%%%%%%%%%%%%%
% % We use newtheorem to define theorem-like structures
% %
% % Here are some common ones. . .
%%%%%%%%%%%%%%%%%%%%%%%%%%%%%%%%%%%%%%%%%%%%%%%%%%%%%%%
\newtheorem{theorem}{Theorem}
\newtheorem{lemma}{Lemma}
\newtheorem{proposition}{Proposition}
\newtheorem{scolium}{Scolium}   %% And a not so common one.
\newtheorem{definition}{Definition}
\newenvironment{proof}{{\sc Proof:}}{~\hfill QED}
\newenvironment{AMS}{}{}
\newenvironment{keywords}{}{}
%%%%%%%%%%%%%%%%%%%%%%%%%%%%%%%%%%%%%%%%%%%%%%%%%%%%%%%
% %   The first thanks indicates your affiliation
% %
% %  Just the name here.
% %
% % Your mailing address goes at the end.
% %
% % \thanks is also how you indicate grant support
% %
%%%%%%%%%%%%%%%%%%%%%%%%%%%%%%%%%%%%%%%%%%%%%%%%%%%%%%%


\begin{document}
\setstretch{1.5} %줄간격 조정
\newpage
\firstpageheader{}{}{\bf\large Daeyoung Lim \\ Grad School \\ Year of 2009, early}
\runningheader{Daeyoung Lim}{Graduate School Pre-exam}{2009 early}
\begin{questions}
   \question
   $X$와 $Y$가 독립적으로 구간 $(0,1)$에서 균일분포(uniform distribution)를 따른다고 하자. 이로부터 $U$와 $V$가 다음과 같이 정의된다.
   $$
    U=XY,\;\; V=\dfrac{X}{Y}
   $$
   \begin{enumerate}
    \item $(U,V)$의 결합밀도(joint density) 함수를 구하여라.
    \item $U$와 $V$ 각각의 주변밀도(marginal density) 함수를 구하여라.
   \end{enumerate}
   \begin{solution}
    \begin{enumerate}
      \item 1개 이상의 변수가 존재하고, 그것들의 변환(transformation)으로 새로운 변수가 정의되었을 때 원래 알던 결합밀도함수에 자코비언만 곱하면 된다. 단, 원래 알던 결합밀도함수의 변수에 새로 정의된 변수를 대입해야 한다. 즉 $U=XY$이고 $V=X/Y$라면 $U,V$의 결합밀도함수는 다음과 같이 정의된다.
      $$
        f_{U,V}\left(u,v\right) = f_{X,Y}\left(x,y\right)\cdot \left|\mathrm{det}\left(J\right)\right|
      $$
      원래 알던 변수 $X,Y$를 새로 정의된 변수 $U,V$로 표현하면 $X=U^{1/2}V^{1/2}$, $Y=U^{1/2}V^{-1/2}$이다. 고로 자코비언은 
      $$
        J=\begin{bmatrix} \dfrac{\sqrt{v}}{2\sqrt{u}}& \dfrac{\sqrt{u}}{2\sqrt{v}} \\ \dfrac{1}{2\sqrt{uv}}& -\dfrac{\sqrt{u}}{2}v^{3/2}  \end{bmatrix}
      $$
    
    이고 $\left|\mathrm{det}\left(J\right)\right|=2^{-1}v^{-1}$이다.
    \begin{align}
      0 &< \sqrt{uv} < 1 \;\; \implies \;\; 0 < u < \dfrac{1}{v}\\
      0 &< \sqrt{\dfrac{u}{v}}<1 \;\; \implies \;\; 0 < u < v
    \end{align}
    중요한 것은 범위인데 우리가 알고 있는 $X$와 $Y$의 범위를 이용하면 된다. 그리고 $u$를 세로축 $v$를 가로축으로 하여 부등식의 범위를 빗금으로 칠하면 그것이 $U$와 $V$가 이루고 있는 범위가 된다. 이 범위는 이중 적분할 때와 같이 한 부등식은 하나가 다른 하나에 의존할 것이고, 나머지 한 부등식은 그 변수가 독립적일 것이다. 그림은 생략하고 그러면 범위가 다음과 같다.
    \begin{align}
      u &< v < \dfrac{1}{u}\\
      0 &< u < 1
    \end{align}
    그러므로 $(U,V)$의 결합밀도함수는
    $$
      f_{U,V}\left(u,v\right) = \dfrac{1}{2v}, \;\; u < v< \dfrac{1}{u},\; 0<u<1
    $$
    이다.
    \item 위에서 구한 범위로 적분해 내면 된다. 다만 독립적이었던 변수를 의존적으로 바꿔주어야 한다. $0<u<v^{-1}$과 $0<u<v$ 둘이 있는데 이에 따라 $v$의 범위가 달라진다. 그림을 그리면 쉽게 알 수 있다.
    \begin{align}
      f_{U}\left(u\right) &= \int_{u}^{1/u}\dfrac{1}{2v}\,dv \\
      &= -\ln u,\;\; 0<u<1\\
      f_{V}\left(v\right) &= \begin{cases} \displaystyle \int_{0}^{v}\dfrac{1}{2v}\,du=\dfrac{1}{2},& 0<v<1 \\\displaystyle \int_{0}^{1/v}\dfrac{1}{2v}\,du=\dfrac{1}{2v^{2}}, & 1 \leq v < \infty \end{cases}
    \end{align}
    \end{enumerate}
   \end{solution}
   \question
   Let $X$ take on the values 0 and 1 with probabilities $p$ and $1-p$, respectively. It is known that $1/3\leq p \leq 2/3$.
   \begin{enumerate}
    \item Find the MLE $\hat{p}$ of $p$.
    \item Find the expected squared error, $\mathrm{E}\left(\hat{p}-p\right)^{2}$ of the MLE.
    \item Show that the expected squared error of the MLE is uniformly larger than that of $\overset{\sim}{p}\equiv 1/2$. That is, $\mathrm{E}\left(\hat{p}-p\right)^{2}>\mathrm{E}\left(\overset{\sim}{p}-p\right)^{2}$ for all $p$.
   \end{enumerate}
   \begin{solution}
    \begin{enumerate}
      \item MLE라는 것이 내가 관측된 샘플을 가장 그럴듯하게 만들어주는 모수의 값을 고르는 것이므로, 만약 $X=1$이라면 $1-p$가 가장 높은 값을 $\hat{p}$로 정할 것이다. 그 값은 알려진 바에 따르면 $1/3$일 것이므로 $\hat{p}=1/3$이다. 반대로 관측된 것이 $X=0$이라면 $p$를 가장 높에 해야 하므로 그 값에 해당하는 $2/3$가 $\hat{p}$일 것이다. 정리하면
      $$
        \hat{p}=\begin{cases}1/3, & \text{if $X=1$ (with prob=$1-p$)} \\ 2/3, & \text{if $X=0$ (with prob=$p$)} \end{cases}
      $$
      \item 위에서 구한 것을 바탕으로
      \begin{align}
        \mathrm{E}\left(\hat{p}-p\right)^{2}&= \mathrm{E}\left(\hat{p}^{2}-2p\hat{p}+p^{2}\right)\\
        &= \dfrac{1}{9}\left(1-p\right)+\dfrac{4}{9}p -2p\left(\dfrac{1}{3}\left(1-p\right)+\dfrac{2}{3}p\right)+p^{2}\\
        &= \dfrac{1}{2}\left(p^{2}-p+\dfrac{1}{3}\right)
      \end{align}
      \item 2에서 이미 구했으므로 $\overset{\sim}{p}=1/2$을 넣어 일단 부등식을 정리하자.
      $$
        \mathrm{E}\left(\hat{p}-p\right)^{2}>\mathrm{E}\left(\overset{\sim}{p}-p\right)^{2} \;\;\implies\;\; -\dfrac{1}{36}\left(24p^{2}-24p+5\right) > 0
      $$ 
      $p$의 범위가 $1/3$과 $2/3$ 사이이므로
      $$
        f\left(p\right) = -\dfrac{1}{36}\left(24p^{2}-24p+5\right)
      $$
      이 최대가 되게 하는 정의역 $p=1/2$를 사이에 두고 양옆으로 걸쳐있다. 즉 $1/3$과 $2/3$에서만 $f\left(p\right)$이 양수면 된다.
      \begin{align}
        f\left(\dfrac{1}{3}\right) &= \dfrac{1}{108}\\
        f\left(\dfrac{2}{3}\right) &= \dfrac{1}{108}
      \end{align}
      사실 위로 볼록이 포물선에서 $1/2$는 $1/3$과 $2/3$의 중간에 위치하므로 값이 같아지는 건 당연하다. 어쨌든 둘 다 양수이므로 증명 끝.
    \end{enumerate}
   \end{solution}
   \question
   다음의 확률 밀도 함수 (probability density function)를 갖는 확률변수를 고려하자.
   $$
    f\left(x;\theta\right)=\theta x^{\theta-1},\;\;0<x<1.
   $$
   \begin{enumerate}
    \item 크기가 1 $(n=1)$인 표본을 이용하여 다음의 가설을 검정하기 위한 $\alpha =0.05$인 최강력 검정법(the most powerful test)을 정의하시오.
    $$
      H_{0}:\theta=1\quad \text{vs}\quad H_{1}:\theta=2
    $$
    \item $\theta=2$인 경우, 1에서 정의된 검정법의 검정력(power)를 계산하시오.
   \end{enumerate}
   \begin{solution}
    \begin{enumerate}
      \item $X$는 $\mathrm{Be}\left(\theta,1\right)$를 따른다. 가능도비를 계산하면
      $$
        \dfrac{L_{0}}{L_{1}} = \dfrac{1}{2x} < k
      $$
      이므로 기각역이 $c<X$이다. 상수 $c$는 $\mathrm{Pr}\left(X>c\;\middle|\;H_{0}\right)=\alpha$를 통해 알 수 있다.
      \begin{align}
        \mathrm{Pr}\left(X>c\;\middle|\;H_{0}\right) &= \int_{c}^{1}1\,dx\\
        &= 1-c = 0.05\\
        c &= 0.95
      \end{align}
      따라서 $X>0.95$일 때 귀무가설을 기각하고 $X\leq 0.95$일 때는 귀무가설을 기각하지 않는 것이 최강력 검정법이다.
      \item 검정력은 대립가설이 참일 때 귀무가설을 기각할 확률이므로 $\mathrm{Pr}\left(X>0.95\;|\;H_{1}\right)$이다.
      \begin{align}
        \mathrm{Pr}\left(X>0.95\;|\;H_{1}\right) &= \int_{0.95}^{1}2x\,dx\\
        &= 1-0.95^{2}\\
        &= 0.0975
      \end{align}
    \end{enumerate}
   \end{solution}
   \question
   다음과 같은 다중선형회귀모형
   $$
    Y=\beta_{0}+\beta_{1}X_{1}+\beta_{2}X_{2}+\beta_{3}X_{3}+\epsilon
   $$
   을 고려하자. 여기서, $Y$는 독립변수를, $X_{1},X_{2},X_{3}$는 설명변수들을, 그리고 $\beta_{0},\beta_{1},\beta_{2},\beta_{3}$는 회귀계수를 의미하며 $\epsilon$은 평균이 0, 분산이 $\sigma^{2}$인 정규분포를 따르는 오차항을 의미한다. 모수에 대한 가설 $H_{0}:\beta_{1}=\beta_{3}$ 대 $H_{1}:\beta_{1}\neq \beta_{3}$를 고려하자. 유의수준 5\% 하에서 이 가설을 검정하는 방법을 자세히 기술하시오.
   \begin{solution}
    자세히 기술하라고 했으므로 정말 자세히 기술하도록 한다. 위와 같은 문제는 회귀계수에 제약조건을 주고 그것이 참인지 가설검정을 하는 법을 묻고 있다. 즉 $H: \mathbf{A}\beta=\mathbf{c}$와 같은 가설을 세운 것이다. 여기서 $\mathbf{A}\\beta$의 각 행이 하나의 제약조건을 구성한다. 검정을 하기 위한 통계량은 당연히 $\mathbf{A}\widehat{\beta}$이 될 것이고 그 값이 $\mathbf{c}$와 너무 다르면 가설을 기각하게 된다. 여기서 너무 `다르면'의 기준을 정해야 하는데 그걸 제공해 주는 것이 바로 `거리' 개념이다. 수학에서 거리 개념은 \emph{distance function} 혹은 \emph{metric}을 통해 계산하게 되는데 집합의 두 원소를 아무렇게나 잡아도 거리를 계산해낼 수 있는 공간을 \emph{metric space}라 부른다. 그리고 두 원소 사이의 거리뿐만 아니라 개별 원소의 `크기'도 계산해낼 수 있으려면 그 공간에 \emph{norm}이 주어져야 한다. 여기서 $\mathbf{A}\widehat{\beta}-\mathbf{c}$의 \emph{norm}을 계산하면 그 거리가 될 것이다. 일반적으로 우리가 쓰는 크기의 개념은 $\ell_{2}$ norm이라 부르고 유한한 벡터공간(finite vector space)에서는 `제곱합'꼴로 표현된다. 즉 벡터로 말하면 \emph{quadratic form}이 되는 것이다.\par
    따라서 $\mathbf{A}\beta-\mathbf{c}$의 제곱합은 $\left(\mathbf{A}\beta-\mathbf{c}\right)'\left(\mathbf{A}\beta-\mathbf{c}\right)$일 텐데 선형회귀에서는 각 추정량의 분산을 고려해주어야 한다. 즉 $\widehat{\beta}$가 결정된 값이 아니라 관측 이전의 값이므로 각각의 원소가 $\mathbf{c}$와의 거리를 계산할 때 얼마나 변하는지 고려하는 것이 상식적일 것이다. 따라서 우리가 정햔 통계량과 검정하고 싶은 값 사이의 거리를 계산하기 위해 도출된 식은 다음과 같다.
    $$
      \left(\mathbf{A}\widehat{\beta}-\mathbf{c}\right)'\left(\mathrm{Var}\left(\mathbf{A}\widehat{\beta}\right)\right)^{-1}\left(\mathbf{A}\widehat{\beta}-\mathbf{c}\right)
    $$
    그리고 다음의 과정을 거쳐서 다시 계산된다.
    \begin{itemize}
      \item $\mathrm{Var}\left(\mathbf{A}X\right)=\mathbf{A}\mathrm{Var}\left(X\right)\mathbf{A}'$ (여기서 $X$는 임의의 random vector)
      \item $\mathrm{Var}\left(\widehat{\beta}\right) = \mathrm{Var}\left(\left(\mathbf{X}'\mathbf{X}\right)^{-1}\mathbf{X}'y\right)=\left(\mathbf{X}'\mathbf{X}\right)^{-1}\mathbf{X}'\sigma^{2}\mathbf{I}_{n}\mathbf{X}\left(\mathbf{X}'\mathbf{X}\right)'=\sigma^{2}\left(\mathbf{X}'\mathbf{X}\right)^{-1}$
      \item 실제로 $\widehat{\beta}$의 분산을 구할 때는 $\sigma^{2}$를 모르므로 $\sigma^{2}$의 비편향추정량인 $S^{2}=\mathrm{SSE}/\left(n-p-1\right)$로 대체한다. 여기서 $p$는 `\emph{predictor}'의 앞글자로 설명변수의 개수를 의미한다. 절편까지 이미 추정한 모수의 개수가 $p+1$이므로 자유도에서 그만큼 빠진다.
    \end{itemize}
    이렇게 되어 우리가 쓰게 될 검정통계량의 초기형태는 다음과 같다.
    $$
      \dfrac{1}{S^{2}}\left(\mathbf{A}\widehat{\beta}-\mathbf{c}\right)'\left(\mathbf{A}\left(\mathbf{X}'\mathbf{X}\right)^{-1}\mathbf{A}'\right)^{-1}\left(\mathbf{A}\widehat{\beta}-\mathbf{c}\right)
    $$
    $H$가 참일 때 구한 $\mathrm{SSE}$을 $\mathrm{SSE}_{H}$이라 표시하고 다변량 정규분포의 특성을 이용하면 다음과 같은 사실이 구해진다.
    \begin{itemize}
      \item $\mathbf{A}\widehat{\beta} \sim \mathcal{N}_{q}\left(\mathbf{c},\sigma^{2}\mathbf{A}\left(\mathbf{X}'\mathbf{X}\right)^{-1}\mathbf{A}'\right)$ (여기서 $q$는 $\mathbf{A}$의 행수)
      \item 위에서 구한 $\mathrm{Var}\left(\mathbf{A}\widehat{\beta}\right)^{-1/2}$를 $\mathbf{A}\widehat{\beta}-\mathbf{c}$ 앞에 곱해주면 $\mathcal{N}_{q}\left(\bs{0},\mathbf{I}_{q}\right)$이 된다.
      \item 즉, 카이제곱분포가 표준정규분포를 따르는 확률변수의 제곱합꼴로 표시된다는 점을 상기해볼 때
      \begin{equation}
        \left(\mathbf{A}\widehat{\beta}-\mathbf{c}\right)'\left(\mathrm{Var}\left(\mathbf{A}\widehat{\beta}\right)\right)^{-1}\left(\mathbf{A}\widehat{\beta}-\mathbf{c}\right) \sim \chi^{2}\left(q\right)
      \end{equation}
      임을 알 수 있다.
      \item 다음과 같은 사실이 성립한다. 증명은 Appendix로 뺀다.
      \begin{equation}
        \dfrac{\mathrm{SSE}_{H}-\mathrm{SSE}}{\sigma^{2}} = \left(\mathbf{A}\widehat{\beta}-\mathbf{c}\right)'\left(\mathrm{Var}\left(\mathbf{A}\widehat{\beta}\right)\right)^{-1}\left(\mathbf{A}\widehat{\beta}-\mathbf{c}\right)
      \end{equation}
      \item 따라서 $\mathrm{SSE}/\sigma^{2}\sim \chi^{2}\left(n-p-1\right)$이므로
      $$
        F = \dfrac{\left(\mathrm{SSE}_{H}-\mathrm{SSE}\right)/\left(\sigma^{2}q\right)}{\mathrm{SSE}/\left(\sigma^{2}\left(n-p-1\right)\right)} \sim F_{q,n-p-1}
      $$
    \end{itemize}
    이제 문제로 다시 돌아가보자. 제약조건은 $\beta_{1}=\beta_{3}$로 한 개이다. 따라서 $\mathbf{A}$의 행은 1개, 즉 $q=1$이다. 따라서
    $$
      \mathbf{A}\beta = \begin{bmatrix}0 & 1 & 0 & -1 \end{bmatrix}\begin{bmatrix}\beta_{0}\\ \beta_{1} \\ \beta_{2} \\ \beta_{3}  \end{bmatrix} = 0
    $$
    이 된다. 그리고
    $$
      F = \dfrac{\mathrm{SSE}_{H}-\mathrm{SSE}/1}{\mathrm{SSE}/\left(n-4\right)}\sim F_{1,n-4}
    $$
    를 따르고 $F>F_{1,n-4}^{-1}\left(0.05\right)$일 때 0.05 신뢰수준에서 귀무가설을 기각한다.
   \end{solution}
   \end{questions}
   \newpage
   \appendix
   \section{$\mathrm{SSE}_{H}$ 유도}
   통계의 많은 문제는 최댓값/최솟값을 찾는 수치최적화 문제로 치환된다. 일반적인 선형회귀 모형은
   $$
    Y=\mathbf{X}\beta+\epsilon
   $$
   이며 위 모형을 적합할 때는
   $$
    \widehat{\beta}= \argmin\limits_{\beta}\left(Y-\mathbf{X}\beta\right)'\left(Y-\mathbf{X}\beta\right)
   $$
   와 같이 최적화 문제로 바뀐다. 일반 선형회귀 모형을 적합하는 것은 제약조건이 없는 경우이지만 $\mathbf{A}\beta=\mathbf{c}$와 같은 제약조건 하에서 최소화하는 $\beta$를 찾는 문제는 라그랑지 승수를 써서 해결한다. 제약조건이 있는 최적화 문제는 다음과 같이 표기한다.
   $$
    \min_{\text{subject to }\mathbf{A}\beta=\mathbf{c}}\left(Y-\mathbf{X}\beta\right)'\left(Y-\mathbf{X}\beta\right)
   $$
   라그랑지 승수를 써서 다시 문제를 쓰면 아래와 같다.
   \begin{align}
    \mathbf{r}&=\left(Y-\mathbf{X}\beta\right)'\left(Y-\mathbf{X}\beta\right)+\bs{\lambda}'\left(\mathbf{A}\beta-\mathbf{c}\right) \\
    \widehat{\beta}_{H}&= \argmin_{\beta} \mathbf{r}
   \end{align}
   라그랑지 승수로 표현된 식을 최소화하는 방법은 정형화되어 있다. 먼저 정의역으로 정의된 변수로 미분을 하고 제약조건에 대입한 뒤에 나온 라그랑지 승수의 해를 원래 정의역으로 정의된 변수 식에 대입하는 것! 그러므로 $\mathbf{r}$을 $\beta$에 대해 미분하면
   $$
    \dfrac{\partial \mathbf{r}}{\partial \beta} = -2\mathbf{X}'Y + 2\mathbf{X}'\mathbf{X}\beta+\mathbf{A}'\bs{\lambda}=\mathbf{0}    
   $$
   이 되고 $\widehat{\beta}_{H}=\widehat{\beta}-\left(\mathbf{X}'\mathbf{X}\right)^{-1}\mathbf{A}'\widehat{\bs{\lambda}}/2$이 도출된다. 이를 원래 제약 조건에 대입하면
   $$
    \widehat{\bs{\lambda}} = -2\left(\mathbf{A}\left(\mathbf{X}'\mathbf{X}\right)^{-1}\mathbf{A}'\right)^{-1}\left(\mathbf{c}-\mathbf{A}\widehat{\beta}\right)
   $$
   이 된다. 이제 $\mathrm{Var}\left(\mathbf{A}\widehat{\beta}\right)^{-1}$꼴이 등장한다. 이를 다시 $\widehat{\beta}_{H}$의 식에 대입하면
   \begin{equation}
    \widehat{\beta}_{H}= \widehat{\beta}+\left(\mathbf{X}'\mathbf{X}\right)^{-1}\mathbf{A}'\left(\mathbf{A}\left(\mathbf{X}'\mathbf{X}\right)^{-1}\mathbf{A}'\right)^{-1}\left(\mathbf{c}-\mathbf{A}\widehat{\beta}\right)
   \end{equation}
   이 나온다. 이제부터 식이 좀 복잡해지는데 다시 $\mathrm{SSE}_{H}-\mathrm{SSE}$를 구하기 위해서는 다음과 같이 시작한다.
   \begin{align}
    \left(Y-\mathbf{X}\widehat{\beta}_{H}\right)'\left(Y-\mathbf{X}\widehat{\beta}_{H}\right) &= \left(Y-\mathbf{X}\widehat{\beta}+\mathbf{X}\widehat{\beta}-\mathbf{X}\widehat{\beta}_{H}\right)'\left(Y-\mathbf{X}\widehat{\beta}+\mathbf{X}\widehat{\beta}-\mathbf{X}\widehat{\beta}_{H}\right)\\
    &= \underbrace{\left(Y-\mathbf{X}\widehat{\beta}\right)'\left(Y-\mathbf{X}\widehat{\beta}\right)}_{\text{SSE}} +\left(\widehat{\beta}-\widehat{\beta}_{H}\right)'\mathbf{X}'\mathbf{X}\left(\widehat{\beta}-\widehat{\beta}_{H}\right)+2\left(Y-\mathbf{X}\widehat{\beta}\right)'\left(\mathbf{X}\widehat{\beta}-\mathbf{X}\widehat{\beta}_{H}\right)\\
    &= \mathrm{SSE}+\left(\widehat{\beta}-\widehat{\beta}_{H}\right)'\mathbf{X}'\mathbf{X}\left(\widehat{\beta}-\widehat{\beta}_{H}\right)+2\underbrace{\left(Y-\mathbf{X}\widehat{\beta}\right)'\mathbf{X}\left(\widehat{\beta}-\widehat{\beta}_{H}\right)}_{=0}\\
    &= \mathrm{SSE}+\left(\widehat{\beta}-\widehat{\beta}_{H}\right)'\mathbf{X}'\mathbf{X}\left(\widehat{\beta}-\widehat{\beta}_{H}\right)
   \end{align}
    여기서 저게 왜 0이 되냐하면 $\mathbf{X}'Y= \mathbf{X}'\mathbf{X}\widehat{\beta}$이기 때문. 이제 (23)의 $\widehat{\beta}$를 넘겨서 (27)에 대입하고 정리하면 샤샤샤 소거되고 (20)이 남는다.
    \section{$\mathrm{SSE}\sim \chi^{2}\left(n-p-1\right)$}
    $Z \sim \mathcal{N}\left(\mathbf{0},\mathbf{I}_{n}\right)$이고 어떤 행렬 $\mathbf{A}$가 $\mathbf{A}^{2}=\mathbf{A}, \mathbf{A}'=\mathbf{A}$를 만족한다고 하자(idempotent). 그러면 다음과 같은 성질을 만족한다.
    \begin{itemize}
      \item $\mathbf{A}$의 고유값(eigenvalue)는 0 또는 1이다.
      \item $Z'\mathbf{A}Z\sim \chi^{2}\left(\Tr\left(\mathbf{A}\right)\right)$이다.
    \end{itemize}
    먼저 고유값이 0 또는 1인 사실은 다음을 통해 간단히 알 수 있다.
    \begin{align}
      \mathbf{A}v &= \lambda v\\
      \mathbf{A}^{2}v &= \lambda \mathbf{A}v\\
      \left(\mathrm{LHS}\right) &= \mathbf{A}v = \lambda v\\
      \left(\mathrm{RHS}\right) &= \lambda^{2}v\\
      \lambda^{2}v &= \lambda v\\
      \therefore \lambda &= 0\text{ or } 1
    \end{align}
    
\end{document}
